\startcontents[chapters]
\chapter{Time Series Analysis}\label{ch:time-series-analysis}
	
\printcontents[chapters]{}{1}{}
\section{Overview of time series analysis}
\subsection{Types of variations}
The source of variations can be classified as: \cite[10]{chatfield2003analysis}
\begin{itemize}
    \item \textbf{seasonal effect}:\cite{wiki:seasonality} When there are patterns that repeat over \textbf{known, fixed periods} of time within the data set it is considered to be seasonality, seasonal variation, periodic variation, or periodic fluctuations. 
    \item \textbf{trend}: long term change in the mean. The 'long' here can be quite surjective.
    \item \textbf{irregular fluctuations}.
\end{itemize}


\subsection{General strategies to transform data for stationarity}
\begin{mdframed}
\begin{enumerate}
\item Transform data so that the residuals are \textbf{stationary}
\begin{itemize}
    \item estimate and subtract trend and seasonal effect. A general trend and seasonal model can be represented as
    $$X_t = T_t + S_t + E_t = \beta_0 + \beta_1 t + \sum_i (m \cos(\lambda_i t) + \gamma_i \sin(\lambda_i t)) + E_t$$
    \item differencing
    \item nonlinear transformation
\end{itemize}
\item Fit model to residuals.
\end{enumerate}
\end{mdframed}


\begin{example} [Use backshifting/differencing to remove trend]
\begin{itemize}
    \item If $X_t = \beta_0 + \beta_1 t + E_t$, then
    $$\nabla X_t = \beta_1 + \nabla E_t$$
    \item If $X_t = \sum_{i=0}\beta_i t^i + E_t$, then
    $$\nabla^k X_t = k! \beta_k + \nabla^k E_t$$
\end{itemize}
\end{example}

\section{Foundations}
\subsection{Time series and its statistics}
\begin{definition}[time series]
\cite[1]{brockwell2002introduction}
A time series is a set of observations, $x_t$, each one being recorded at a specific time. A \textbf{discrete time series} is one in which the indexing set is a discrete set. A \textbf{continuous time series} is one in which the indexing set is an interval on the real line. 
\end{definition}

\begin{definition}[time series model]
\cite[7]{brockwell2002introduction} A time series model for the observed data $\{x_t\}$ is a specification of the joint distributions(or possibly only the means and covariance) of a sequence of random variables $\{X_t\}$ of which $\{x_t\}$ is postulated to be a realization.
\end{definition}

\begin{remark}
A complete model is proposing a parametric form of 
$$P(X_1\leq x_1,...,X_n\leq x_n)$$
which however,contains far too many parameters to estimated; Therefore, we usually only specify lower order moments of the joint distribution, such as $E[X_t],E[X_tX_{t+h}]$.
\end{remark}

\begin{definition}[mean and covariance function]
Let $\{X_t\}$ be a discrete time series with $E[X_t^2] < \infty$. The mean function of $\{X_t\}$ is $$\mu_X(t) = E[X_t]$$
The covariance function of $\{X_t\}$ is
$$\gamma_X(r,s) = Cov(X_r,X_s) = E[(X_r-\mu_X(r))(X_s-\mu_X(s))]$$
for all integer $r,s$.
\end{definition}


\begin{lemma}[Basic property of $\gamma(h)$]
\cite[47]{brockwell2002introduction}For a stationary $\{X_t\}$, its autocovariance function
\begin{itemize}
    \item $\gamma(0) \geq 0$
    \item $\abs{\gamma(h)}\leq \gamma(0),\forall h$
    \item $\gamma(h)$  is even, i.e., $\gamma(h) = \gamma(-h), \forall h$.
\end{itemize}
\end{lemma}
\begin{proof}
	(1) It is variance. (2) From Cauchy inequality(\autoref{ch:theory-of-probability:th:Cauchy-SchwarzInequalityBoundsOnCorrelations}). (3) $\gamma(h) = Cov(X_t,X_{t+h}) = Cov(X_{t+h},X_t) = \gamma(-h)$.
\end{proof}





\begin{definition}[non-negative definite function]
\cite[46]{brockwell2002introduction}
A real-valued function $k$ defined on integers is non-negative definite if
$$\sum_{i,j=1}^n a_i k(i-j)a_j \geq 0$$
for all positive integers $n$ and vectors $a=(a_1,a_2,...,a_n)$.
\end{definition}


\begin{theorem}[non-negative of covariance matrix]
\cite[47]{brockwell2002introduction}A real-valued function defined on the integers is the autocovariance function of a stationary time series if and only if it is even and non-negative definite.
\end{theorem}
Proof: (1) Let $a$ be any vector of size $n$, let $X=(X_1,X_2,...,X_n)^T$ The covariance matrix of random variable $a^TX$ is $a^TCov(X)a = Var(a^TX) = \sum_{i,j=1}^n a_i \gamma(i-j)a_j \geq 0$, where we have use the facts (1) $Cov(X)_{i,j} = \gamma(i-j)$; (2) The transformation rule for covariance matrix transformation for multi-dimensional random vectors. (2) The other direction see reference. 


\begin{lemma}[demean operation for stationary process]
Given a stationary process $X_t$ with mean $\mu$ and autocovariance function $\gamma_X$, then the process
	$$Y_t = X_t-\mu$$ has mean zero and same autocovariance function.
\end{lemma}
\begin{proof}
	Straight forward.
\end{proof}
\subsection{Stationarity}

\begin{definition}[weakly stationary]
	\cite[15]{brockwell2002introduction}$\{X_t\}$ is weakly stationary if 
	\begin{itemize}
		\item $\mu_X(t)$ is independent of $t$.
		\item $\gamma_X(t+h,t)$ is independent of $t$ for each $h$.
	\end{itemize}
\end{definition}



\begin{remark}
	A stationary process will not contain periodic/seasonal change and trends.
\end{remark}

\begin{definition}[strictly stationary]
	\cite[49]{brockwell2002introduction} $\{X_t\}$ is a strictly stationary time series if the joint distributions of $(X_1,...,X_n)$ and joint distribution of $(X_{1+h},...,X_{n+h})$ are equal for all integers $h$ and $n \geq 1$. Or we write
	$$P(X_1,...,X_n) = P(X_{1+h},...,X_{n+h})$$
\end{definition}

\begin{lemma}[properties of strictly stationary process]
	\cite[49]{brockwell2002introduction} Let $\{X_t\}$ be a strictly stationary time series, then we have
	\begin{enumerate}
		\item The random variable $X_t$ are identically distributed;
		\item $P(X_t,X_{t+h}) = P(X_1,X_{1+h}$ for all integers $t$ and $h$;
		\item $\{X_t\}$ is weakly stationary if $E[X_t^2] < \infty $ for all $t$;
		\item Weak stationarity does not imply strict stationarity
		\item All iid sequence is strictly stationary.
	\end{enumerate}
\end{lemma}
Proof: (1)(2)(3) Directly from definition, joint distribution equal implies marginal distribution equal.

\begin{remark}
	A stationary process will not contain trends and periodic trends.
\end{remark}


\begin{definition}
	Let $\{X_t\}$ be a stationary time series. The autocovariance function of $\{X_t\}$ at lag $h$ is
	$$\gamma_X(h) = Cov(X_{t+h},X_t)$$
	The autocorrelation function of $\{X_t\}$ at lag $h$ is
	$$\rho_X(h) = \frac{Cov(X(h),X(0)}{\sqrt{Cov(X(h),X(h)}\sqrt{Cov(X(0),X(0))}} =  \frac{\gamma_X(h)}{\gamma_X(0)} = Cor(X_{t+h},X_t)$$
\end{definition}


\subsection{Partial autocorrelation function}\label{ch:time-series-analysis:sec:PartialAutocorrelationFunction}



\begin{definition}[partial autocorrelation function, PACF]\cite[100]{subbaRao2017timeSeries}
Let $\{X_t\}$ be a zero mean stationary process. The partial autocorrelation at lag $h$ for $h\geq 1$, denoted by $\pi_X(h)$, is defined as the direct correlation between $X_t$ and $X_{t+h}$ with the \textbf{linear dependence} between the intermediate variables $X_s, t<s<t+h$ removed. Specifically, we have  
	$$PACH(X_t,X_{t+h}) = Cov(X_t - Var[Y]^{-1}E[YX_i]Y,X_{t+h}-Var[Y]^{-1}E[YX_{t+h}]]Y),$$
where $Y = (X_{t+1},X_{t+2},...,X_{t+h-1}).$	
\end{definition}




\begin{theorem}[computation of PACF]\label{ch:time-series-analysis:th:ComputePACFViaLinearRegression}
	Let $\{X_t\}$ be a zero-mean stationary process.
\begin{itemize}
	\item The partial autocorrelation at lag $h$ for $h\geq 2$, denoted by $\pi_X(h)$, is equal to the coefficient $a_h$ from the optimal linear prediction of $X_{t+h}$ on the observations $X_{t}, X_{t+1},...,X_{t+h-1}$ given by
	$$X_{t+h} = a_h X_t + \beta_1 X_{t+1} + ... + \beta_{h-1} X_{t+h-1}.$$
	For $h=0$, $\pi_X(0) = 1$;$h = 1$, $\pi_X(1) = \rho_X(1)$.
	\item The partial autocorrelation at lag $h$ for $h\geq 2$, denoted by $\pi_X(h)$, is given by the partial correlation conditioned on $(X_{t+1},X_{t+2},...,X_{t+h-1})$,i.e.,
	\begin{align*}
	\pi_X(h) &= corr(X_{t+h} - P_Y(X_{t+h}),X_t - P_Y(X_{t})) \\
	&= \frac{Cov(X_{t+h} - P_Y(X_{t+h}),X_t - P_Y(X_{t}))}{\sqrt{Var[(X_{t+h} - P_Y(X_{t+h}))]}\sqrt{Var[X_t - P_Y(X_{t})]}}
	\end{align*}
where $Y = (X_{t+1},X_{t+2},...,X_{t+h-1})$, and $P_Y(X_{t+h})$ is the projection of $X_{t+h}$ onto the subspace spanned by $(X_{t+1},X_{t+2},...,X_{t+h-1})$.	
\end{itemize}	
\end{theorem}
\begin{proof}
From the Hilbert space approximation theory(\autoref{ch:functional-analysis:th:normalequation}) and best linear predictor theory(\autoref{}), we know that the coefficient $a_h$ is given by
$$\pi_X(h) = \frac{Cov(X_{t+h} - P_Y(X_{t+h}),X_t - P_Y(X_{t}))}{\sqrt{Var[(X_{t} - P_Y(X_{t}))]}\sqrt{Var[X_t - P_Y(X_{t})]}}.$$
From the definition partial correlation coefficient(\autoref{}), we have
$$	\pi_X(h) = \frac{Cov(X_{t+h} - P_Y(X_{t+h}),X_t - P_Y(X_{t}))}{\sqrt{Var[(X_{t+h} - P_Y(X_{t+h}))]}\sqrt{Var[X_t - P_Y(X_{t})]}}.$$

To show these two are equivalent, our goal is to show
$Var[(X_{t} - P_Y(X_{t}))]$
and $Var[(X_{t+h} - P_Y(X_{t+h}))]$
are equivalent. We note that (\autoref{}) $Var[(X_{t} - P_Y(X_{t}))]$ is fully determined by $Var[X(t)]$, the covariance structure $Cov[Y]$, and the covariance structure of $Cov[Y,X(t)]$. Due to the weak stationarity, $Var[X(t)] = Var[X(t+h)]$, the $Cov[Y,X(t)], Cov[Y,X(t+h)]$ only differ at the labeling. 
\end{proof}


\begin{note}[caution on calculating PACF and its application]\hfill
\begin{itemize}
	\item Suppose we have stationary time series given by
	$$X_{t+1} = \beta_0 X_t + \beta_1 X_{t-1} + \beta_2 X_{t-2} + ... + \beta_q X_{t-q} Z_t,$$
	when we use linear regression model	to compute 
	$$X_{t+h} = a_h X_t + a_1 X_{t+1} + ... + a_{h-1} X_{t+h-1}.$$
	where $1<h \leq q$, only when $h = q$, the coefficient $\pi(h) = a_h = \beta_q$; when $h > q$, $a_h = 0$, when $h < q$,  $a_h$ is not equivalent to $\beta_h$.
	\item Therefore, PACF $\pi(h)$ is the method to estimate the order of AR process; it cannot be used to estimate the coefficient in the AR model; The estimation of the coefficient usually rely on maximum likelihood or least square.  
\end{itemize}	
	
 
\end{note}




\begin{lemma}[bounded variance of sample partial autocorrelation function]\cite[66]{box2015time}\hfill
\begin{itemize}
	\item The partial autocorrelation function of leg $h$, denoted by $\hat{\pi}(h)$, with for a single random sample,  can be viewed as a random variable with support $[-1,1]$ and bounded variance of $Var[\hat{\pi}] \leq 1$. 
	\item If $n$ is the number of samples used in the calculation of PACF $\hat{\pi}$, then   $Var[\hat{\pi}] \leq 1/n$.
\end{itemize}	
	
\end{lemma}
\begin{proof}
	(1) $\abs{\hat{\rho}} \leq 1$ is showed in Cauchy inequality(\autoref{ch:theory-of-probability:th:Cauchy-SchwarzInequalityBoundsOnCorrelations}).
	(2) The boundedness the variance is at
	\autoref{ch:theory-of-probability:th:PopoviciuUpperBoundForVariance}.
\end{proof}


\begin{remark}[interpretation of PACF as conditional correlation]\hfill
	\begin{itemize}
		\item Consider a regression context in which $y$ is the response variable and $x_1,x_2,x_3$ as predictor variables. The partial correlation between $y$ and $x_3$ is given by
		$$\frac{Cov(y,x_3|x_1,x_2)}{\sqrt{Var(y|x_1,x_2)}\sqrt{Var(x_3|x_1,x_2)}}.$$
		\item We can find out the partial correlation between $y$ and $x_3$ by constructing an optimal linear regression
		$$y = \beta_1 x_1 + \beta_2 x_2 + \beta_3 x_3,$$
		and take $\beta_3$ as the PCAF.
	\end{itemize}	
\end{remark}


\begin{lemma}[PACF for AR and MA process]\label{ch:time-series-analysis:th:PACFforARandMA}\hfill
\begin{itemize}
	\item Consider a AR(p) process given by
	$$X_t = a_1 X_{t-1} + a_2 X_{t-2} + ... + a_p X_{t-p} + Z_t, Z_t \sim WN(0, \sigma^2),$$
	then
	$$\pi_X(h) = \begin{cases*}
	a_h, h = 1,2,..,p \\
	0, otherwise
	\end{cases*}.$$
	\item Consider a MA(1) process given by
	$$X_t = Z_t + \theta_1 Z_{t-1}, Z_t \sim WN(0, \sigma^2),$$
	then
	$$\pi_X(h) = \begin{cases*}
	a_h, h = 1,2,..,p \\
	0, otherwise
	\end{cases*}.$$
	
\end{itemize}	
	
\end{lemma}
\begin{proof}
(1)From \autoref{ch:time-series-analysis:th:ARRerepresentationMA(1)Process} directly.
(2)\autoref{ch:time-series-analysis:th:ARRerepresentationMA(1)Process}
\end{proof}

\begin{note}[ACF vs. PACF for application in order identification]\hfill
\begin{itemize}
	\item PACF is useful in identifying the order of AR process; however, ACF has nonzero coefficient for all lags.
	\item ACF is useful in identifying the order of MA process; however, PACF has nonzero coefficient for all lags.
\end{itemize}	
\end{note}


\subsection{Statistical tests}
\subsubsection{Common statistics}
\begin{definition}
	\cite[19]{brockwell2002introduction} Let $x_1,...,x_n$ be observations of a time series. The \textbf{sample mean} is
	$$\bar{x} = \frac{1}{n}\sum_{i=1}^nx_i$$
	The \textbf{sample autovariance function} is
	$$\hat{\gamma}(h) = \frac{1}{n}\sum_{i=1}^n(x_i-\bar{x})( x_{i+h} - \bar{x})$$
	The \textbf{sample autocorrelation function} is
	$$\hat{\rho}(h)=\hat{\gamma}(h)/\hat{\gamma}(0)$$
\end{definition}

\begin{remark}
	Note that the divisor in the covariance is 1/n.
\end{remark}


\begin{lemma}[bounded variance of sample autocorrelation function]\label{ch:time-series-analysis:th:BoundedVarianceOfSampleAutocorrelationFunction}\hfill
\begin{itemize}
	\item The sample autocorrelation function of leg $h$ defined by $\hat{\rho}(h)=\hat{\gamma}(h)/\hat{\gamma}(0)$ is random variable with support $[-1,1]$ and bounded variance of $Var[\hat{\rho}] \leq 1$.
	\item If $n$ is the number of samples used in the calculation of ACF $\hat{\rho}$, then   $Var[\hat{\rho}] \leq 1/n$. 
\end{itemize}	
	
\end{lemma}
\begin{proof}
(1) $\abs{\hat{\rho}} \leq 1$ is showed in Cauchy inequality(\autoref{ch:theory-of-probability:th:Cauchy-SchwarzInequalityBoundsOnCorrelations}).
(2) The boundedness the variance is at
\autoref{ch:theory-of-probability:th:PopoviciuUpperBoundForVariance}.
\end{proof}


\subsubsection{Rolling analysis}

\begin{remark}[motivation for rolling analysis]
	Consider the analysis of a univariate time series $y_t$ over a sample from $t = 1,...,T$. We want to assess \textbf{parameter(mean, variance) constancy over the entire sample} let $n$ denote the 	
\end{remark}

\begin{definition}
	Consider the analysis of a univariate time series $y_t$ over a sample from $t=1,2,...,T$. Let $n$ denote the width of a sub-sample or window and defien the \textbf{rolling} sample means, variances and standard deviation 
	\begin{align*}
	\hat{\mu}_t(n) &=\frac{1}{n}\sum_{i=0}^{n-1} y_{t-i}\\
	\hat{\sigma}^2_t(n) &=\frac{1}{n}\sum_{i=0}^{n-1} (y_{t-i} - \hat{\mu}_t(n))^2\\
	\hat{\sigma}_t(n) &=\sqrt{\hat{\sigma}^2_t(n)}
	\end{align*}	
\end{definition}


\begin{definition}[exponentially weighted moving averages(EWMA)]\index{exponentially weighted moving averages} An $n$ period EWMA of a time series $y_t$ is defined as
	$$\hat{\mu}_t(n) =\sum_{i=0}^{n-1} w_i y_{t-i}, w_i = \frac{\lambda^{i-1}}{\sum_{i=0}^{n-1}\lambda^{i-1}},$$
	where $0<\lambda < 1$ is the decay parameter.	
\end{definition}

\begin{remark}
	In the EWMA, we put more weight on the most recent observations.
\end{remark}

\subsubsection{Test for correlation}
\begin{definition}[Ljung and Box serial correlation test]\cite[32]{tsay2005analysis}\label{ch:time-series-analysis:def:LjungBoxSerialCorrelationTest}
Let $\rho_i$ be the autocorrelation function of a time series $\{X_t\}$. Consider the null hypothesis and alternative hypothesis
\begin{itemize}
	\item null hypothesis $H_0: \rho_1 = \cdots = \rho_m = 0.$
	\item alternative hypothesis $H_1: \rho_i\neq 0, for~some~ i\in \{1,2,...,m\}$.
\end{itemize}
 The test statistic for the hypothesis is given by	
	$$Q(m) = T(T+2)\sum_{i=1}^m \frac{\hat{\rho}_l^2}{T-l}.$$

The decision rule is to reject $H_0$ if $Q(m) > \chi^2_\alpha$(where $\chi^2_\alpha$ denote the $\alpha$ percentile of a chi-squared distribution). 	
\end{definition}

\begin{remark}[interpretation]
If there exists correlation, $Q(m)$ tends to be large.	
\end{remark}


\subsection{Best linear predictor}
\begin{definition}[linear predictor]
	The linear predictor for $X_{n+h}$ given $X_1,X_2,...,X_n$ is given as
	$$P_nX_{n+h} = a_0 + a_1X_1 + ... + a_nX_n$$
\end{definition}

\begin{mdframed}
	\textbf{Determine coefficients for linear predictor} with known $\mu$ and $\gamma$ for stationary process $X_t$ \cite[62]{brockwell2002introduction}\\
	\begin{itemize}
		\item Our goal is to find
		$$a = \arg\min E[(X_{n+h} - P_nX_{n+h})^2]$$
		\item Solve this optimization problem, we have: 
		$$a_0 = \mu(1 - \sum_{i=1}^n a_i)$$
		$$ a = \Gamma^{-1} \gamma_n(h)$$
		where $a=(a_1,a_2,...,a_n)^T$, $\Gamma$ is the matrix with $\Gamma_{i,j} = \gamma(i-j)$, $\gamma_n(h) = (\gamma(h), \gamma(h+1),...,\gamma(n+h-1))$
	\end{itemize}
\end{mdframed}

\begin{lemma}[properties of best linear predictor]
	The predictor $P_nX_{n+h}$ will satisfy:
	\begin{itemize}
		\item $P_nX_{n+h} = \mu + \sum_{i=1}^n a_i(X_i - \mu)$
		\item $E[(X_{n+h}-P_nX_{n+h})^2] = \gamma(0)-a^T\gamma_n(h)$
		\item $E[X_{n+h} - P_nX_{n+h}] = 0$
		\item $E[(X_{n+h} - P_nX_{n+h})X_j] = 0, j=1,2,...,n$
	\end{itemize}
\end{lemma}
\begin{proof}

\end{proof}


\begin{remark}[interpretation]\hfill
	\begin{itemize}
		\item The variance reduction is important, because $a^T\gamma_n(h)$ is non-negative, which suggest that if more previous data (correlated) $X_i$ are use, the more uncertainty is reduced. 
		\item If $\gamma(h) = 0,\forall h>0$, then the best prediction is $\mu$, with variance $\gamma(0)$; that is, no previous information can be exploited. For example, we cannot predict white noise.
	\end{itemize}
\end{remark}

\section{The lag operator and polynomial}
\begin{definition}[lag operator and lag polynomial]\index{lag operator}\index{lag polynomial}
A lag operator is defined as $Lx_t = x_{t-1}$. A lag polynomial is defined as
$$P(L) = a_0 + a_1L + a_2L^2 + ... + a_pL^p.$$
An inverse of a lag polynomial is defined as as$$P^{-1}(L)P(L)=P(L)P^{-1}(L)=1.$$
\end{definition}

\begin{remark}\hfill
\begin{itemize}
    \item The lag operator can operate(addition,subtraction,multiplication, division) like integers.
\end{itemize}
\end{remark}

\begin{lemma}[inverse of factor]\label{ch:time-series-analysis:th:inverselagpolynomialfactor}
The inverse of $P(L)=1-aL,a\in \C$ exists when $\abs{a}< 1$and the inverse is given as
$$ P^{-1}(L) = 1 + aL +a^2L^2 + a^3L^3 + ... $$
\end{lemma}
\begin{proof}
	directly use the definition to verify $P(L)P^{-1}(L)=1$ based on the fact of $a^\infty = 0$.
\end{proof}

\begin{lemma}[factoring lag polynomial]
Any lag polynomial of degree $q$ can be written as
$$P(L) = (1-\beta_1 L)(1-\beta_2 L)...(1-\beta_q L)$$
where $\beta_1,\beta_2,...,\beta_q \in \C$ are the $q$ roots of $P(x)=0$.
\end{lemma}
\begin{proof}
Directly from fundamental theorem of algebra \autoref{ch:topics-in-abstract-algebra:th:fundamentalthalgebra}
\end{proof}


\begin{lemma}[inverse of general lag polynomial]\label{ch:time-series-analysis:th:inverselagpolynomial}
A lag polynomial $P(L)$ of degree $q$ has its inverse if all its roots $\beta_i\in \C$ satisfying 
$$\abs{\beta_i} < 1,i=1,...,q$$
and the inverse is given as:
 $$P^{-1}(L)=(1-\beta_1 L)^{-1}...(1-\beta_q L)^{-1}$$
where $\beta_1,\beta_2,...,\beta_q$ are the $q$ roots of $P(x)=0$.
\end{lemma}
\begin{proof}
	Based on the assumption, each factor exists and can be inverted(
 \autoref{ch:time-series-analysis:th:inverselagpolynomialfactor})). Then the finite product of these factor will exist.
\end{proof}


\begin{definition}[difference operator]
The difference operator $\Delta$ is $1-L$, i.e., $\Delta x_t = x_t - x_{t-1}$. $\Delta^2 = (1-L)^2 = 1-2L+L^2$,i.e., $\Delta^2 x_t = x_t - 2x_{t-1} + x_{t-2}$
\end{definition}

\begin{example}
Let $y_t = \nabla x_t = (1-L)x_t$, we can solve $x_t = (1-L)^{-1}y_t = \sum_{-\infty}^t y_{i}$, or $x_t - x_0 = \sum_{i=1}^t y_{i}$
\end{example}

\section{Linear stationary process}
\subsection{Linear process}
\begin{definition}[linear process]
\cite[51]{brockwell2002introduction} The time series $\{X_t\}$ is a linear process if it has the representation 
$$X_t = \mu + \sum_{j=-\infty}^\infty \phi_j Z_{t-j}$$
for all $t$, where $\{Z_t\}~WN(0,\sigma^2)$ and $\{\phi_j\}$ is a sequence of constants with absolute summability $\sum_{j=-\infty}^\infty \abs{\phi_j} < \infty$.
\end{definition}

\begin{lemma}[basic property of linear process, stationarity]\label{ch:time-series-analysis:th:BasicPropertyOfLinearProcess}
A linear process $$X_t = \mu + \sum_{j=-\infty}^\infty \phi_j Z_{t-j}$$
with  $\sum_{j=-\infty}^\infty \abs{\phi_j} < \infty$ has the following properties:	
\begin{itemize}
	\item $E[X_t] = \mu$.
	\item stationary covariance $$Cov[X_t,X_{t+\tau}] = \sigma^2\sum_{i = -\infty}^{\infty} \phi_i\phi_{i-\tau}.$$
\end{itemize}

That is, a linear process is weakly stationary.
\end{lemma}
\begin{proof}
(1) We are able to exchange the summation and the integral due to \autoref{ch:calculus:th:ExchangeSumAndLebesgueIntegral}. To apply this theorem, note that $E[\abs{Z_t}] = \sigma\sqrt{2/\pi}$ and  $\sum_{j=-\infty}^\infty \sigma\sqrt{2/\pi}\abs{\phi_j} < \infty.$
(2)The convergence of $\sum_{i = -\infty}^{\infty} \phi_i\phi_{i-\tau}$ is discussed in \autoref{ch:sequences-series:th:importantseriesconvergenceresult}
\end{proof}

\begin{definition}[causality]
A linear process $\{X_t\}$ is causal(strictly, a causal function of $\{W_t\}$) if there exists a 
$$\phi(B) = \phi_0 + \phi_1 B + \phi_2 B^2 + ...$$
with $\sum_{i=1}^\infty \abs{\phi_i} < \infty$ and $X_t = \phi(B)W_t$, where $B$ is the lag operator.
\end{definition}

\begin{definition}[invertibility]\label{ch:time-series-analysis:def:invertibility}
A linear process $\{X_t\}$ is invertible(strictly, an invertible function of $\{W_t\}$) if there exists a $$\pi(B) = \pi_0 + \pi_1 B + \pi_2 B^2 + ...$$
with $\sum_{i=1}^\infty \abs{\phi_i} < \infty$ and $W_t = \phi(B)X_t$, where $B$ is the lag operator.
\end{definition}





\begin{theorem}[linear combination of stationary process is stationary]\label{ch:time-series-analysis:th:linearcombinationstationary}\cite[52]{brockwell2002introduction}
Let $Y_t$ be a stationary time series with mean 0 and covariance function $\gamma_Y$. If $\sum_{j=-\infty}^\infty \abs{\psi_j} < \infty$, then the time series
$$X_t = \sum_{j=-\infty}^\infty \psi_j Y_{t-j} = \psi(L)Y_t$$
is stationary with mean 0 and autocovariance function
$$\gamma_X(h) = \sum_{j=-\infty}^\infty\sum_{k=-\infty}^\infty \psi_j\psi_k\gamma_Y(h+k-j)$$
In particular, if $Y(t)$ is the white noise process, then
$$\gamma_X(h) = \sum_{j=-\infty}^\infty \psi_j\psi_{j+h}\sigma^2$$
\end{theorem}
\begin{proof}
(1) We are able to exchange the summation and the integral due to \autoref{ch:calculus:th:ExchangeSumAndLebesgueIntegral}. To apply this theorem, note that $E[\abs{Y_t}] = 0$ and  $\sum_{j=-\infty}^\infty \abs{\psi_j} < \infty.$
(2)
(3) \autoref{ch:time-series-analysis:th:BasicPropertyOfLinearProcess}.
\end{proof}

\subsection{MA processes}
\subsubsection{Basics}
\begin{definition}[moving average operator of order $q$]
\cite[10]{box2015time} A moving average operator of order $q$ can be defined as
$$\theta_q(B) = 1+\theta_1B+\theta_2B+...+\theta_qB^q.$$
\end{definition}


\begin{definition}[MA(q) process]
\cite[50]{brockwell2002introduction}
A time series $\{X_t\}$ is a moving-average process of order $q$ if
$$X_t = Z_t + \theta_1 Z_{t-1} + \theta_2 Z_{t-2} + ... + \theta_q Z_{t-q}$$
where $\{Z_t\}\sim WN(0,\sigma^2)$, and $\theta_1,\theta_2,...,\theta_q$ are constants.
\end{definition}



\begin{theorem}[basic statistics of MA(q)]\cite[33]{chatfield2003analysis}
Given a MA process of order $q$,
$$X_t = Z_t + \theta_1 Z_{t-1} + \theta_2 Z_{t-2} + ... + \theta_q Z_{t-q},$$
 we have 
$$E[X_t] = 0$$
$$Var(X_t) = \sigma^2 \sum_{i=0}^q \theta_i^2$$
and
$$\gamma(k) = \begin{cases} 
0,k>q\\
\sigma^2 \sum_{i=0}^{q-k}\theta_i\theta_{i+k},k=0,1,...,q\\
\gamma(-k),k<0
\end{cases},$$
and
$$\rho(k) = \begin{cases} 
1, k=0\\
0,k>q\\
\sigma^2 \sum_{i=0}^{q-k}\theta_i\theta_{i+k}/\sum_{i=0}^q \theta_i^2,k=1,...,q\\
\gamma(-k),k<0
\end{cases},$$
where $\theta_0 = 1$ is used.
\end{theorem}
\begin{proof}
(1) 
$$E[X_t] = E[Z_t + \theta_1 Z_{t-1} + ... + \theta_q Z_{t-q}] = E[Z_t] + \theta_1 E[Z_{t-1}] +  ... + \theta_q E[Z_{t-q}] = 0.$$
(2) $$Var[X_t] = Var[Z_t] + Var[\theta_1Z_{t-1}] + ... + Var[\theta_q Z_{t-q}] + .cross.terms = \sigma^2+\theta_1\sigma^2 + ... + \theta_q^2\sigma^2.$$
(3) $$Cov(X_t,X_{t+k}) = Cov(Z_t + \theta_1 Z_{t-1} + ... + \theta_q Z_{t-q},Z_t + \theta_1 Z_{t-1 + k} + ... + \theta_q Z_{t-q + k})$$
\end{proof}

\begin{corollary}[basic statistics of MA(1)]
Given a MA process of order $1$,
$$X_t = Z_t + \theta_1 Z_{t-1},$$
we have 
$$E[X_t] = 0$$
$$Var(X_t) = \sigma^2(1 + \theta_1) $$
$$\gamma(1) = \sigma^2 \theta_1$$
$$\rho(1) = \theta_1/(1 + \theta_1^2)$$
\end{corollary}


\begin{theorem}[MA Representation Theorem]\index{MA Representation Theorem}
\cite[50]{brockwell2002introduction} If $\{X_t\}$ is a stationary $q$ correlated time series with mean $0$, then it can be represented as $MA(q)$ process.
\end{theorem}
\begin{proof}
First, An $MA(q)$ process has exactly $q+1$ unknown parameters ($\sigma,\theta_1,...,\theta_q$), which can be determined from the autocovariance function of the given $q$ correlated time series. 
Note that we can first uniquely determine $\theta_q$ by matching $\gamma(q)$, and then $\theta_{q-1}$, and so on.  
\end{proof}


\begin{lemma}[sufficient condition for equivalent AR representation]\label{ch:time-series-analysis:th:ARRerepresentationMA(1)Process}
An $MA(1)$ process $X_t = W_t + \theta W_{t-1}$ can be represented as an AR process of infinite terms: 
$$X_t = \sum_{j=1}^\infty -(-\theta)^j X_{t-j} + W_t$$
if $\abs{\theta}<1$.
\end{lemma}
\begin{proof}
$$X_t = W_t + \theta W_{t-1} = (1+\theta B)W_t$$
then if $\abs{\theta} < 1$, 
we have
$$W_t = (1+\theta B)^{-1}X_t = \sum_{j=0}^\infty (-\theta)^j B^j X_t$$
\end{proof}


\subsubsection{Stationarity and invertibility}
\begin{lemma}[invertibility of MA]
An $MA(q)$ process given as
$$X_t = \theta_q(B)W_t$$
is invertible, if $\theta_q(x) \neq 0, \forall \abs{x} \leq 1$. That is, the polynomial $\theta_q(x)$ having all root lying outside unit circile.
\end{lemma}

\begin{remark}[invertibility and uniqueness]
Consider $X_t = W_t + 0.5W_{t-1},W_t \sim WN(0,25)$ and $Y_t = V_t+5V_{t-1},V_t \sim WN(0,1)$
are the same, but one is invertible, but one is not.
\end{remark}


\begin{lemma}[MA process is always weakly stationary]\label{ch:time-series-analysis:th:MAprocessisstationary}\cite[59]{tsay2005analysis}
The $MA$ process is stationary for any value of $\theta_1,\theta_2,...,\theta_q$.
\end{lemma}
\begin{proof}
See linear combination of white noise process is still stationary process \autoref{ch:time-series-analysis:th:linearcombinationstationary}. Note that any MA process can be viewed as a linear combination of stationary process $Z_1,Z_2,...$
\end{proof}


\begin{remark}
For an MA process, we are not concerned with its stationarity property but concerned with its invertibility.
\end{remark}

\subsubsection{Identification and parameter estimation}


\begin{lemma}[order identification for MA processes]
Suppose we have random sample from the MA time series $\{X_t\},t=1,2,...,n$. The  ACF of order $h$ is zero with 95\% confindence level if
$$\abs{\hat{\rho}(h)} \leq \frac{2}{\sqrt{n}}.$$		
\end{lemma}
\begin{proof}
Consider the null hypothesis that $\rho(h) = 0$. Note that the calculation $\hat{\rho}$ is given by

$$\hat{\rho}(h)=\hat{\gamma}(h)/\hat{\gamma}(0)$$
The \textbf{sample autovariance function} is
 $$\hat{\gamma}(h) = \frac{1}{n}\sum_{i=1}^n(x_i-\bar{x})( x_{i+h} - \bar{x}).$$
The null hypothesis ensures that each term $(x_i-\bar{x})( x_{i+h} - \bar{x})$ is independent and the Cauchy inequality ensures that it is random variable with support $[-1,1]$, mean 0, and variance $\sigma^2$ bounded by $1$.

From central limit theorem(\autoref{ch:theory-of-probability:centralLimitTheorem}), $\hat{\rho} \sim N(0,\frac{\sigma^2}{n}).$ 
Take 95\% confidence level, when $$\abs{\hat{\rho}} \geq \frac{2}{\sqrt{n}}\geq \frac{2\sigma}{\sqrt{n}},$$ we reject the null hypothesis. 
\end{proof}

\begin{remark}[order identification]
	We cannot use the PACF to identify the order of MA process, because PACF has non-zero value extending to infinity\autoref{ch:time-series-analysis:th:PACFforARandMA}.
\end{remark}

\begin{definition}[maximum likelihood estimation of the coefficients in invertible MA(q)]Consider an invertible $MA(q)$ 
	$$X_t = \mu + Z_t + \beta_1 Z_{t-1} + \beta_2Z_{t-2}+\cdots + \beta_q Z_{t-q},$$
	where $Z_t\sim WN(0, \sigma^2)$.
	
	Let $(X_1,X_2,...,X_n)$ be the observations. Then the MLE is given by
	$$L(\mu, \beta_1,...\beta_q, \sigma^2) = \frac{1}{\sqrt{(2\pi)^n \abs{\Sigma}}} \exp(-\frac{1}{2}(X-\mu)^T\Sigma^{-1}(X-\mu)),$$
	where  	
	$$\Sigma_{ij} = Var[X_t]\rho(i-j),\rho(k) = \begin{cases} 
	1, k=0\\
	0,k>q\\
	\sigma^2 \sum_{i=0}^{q-k}\beta_i\beta_{i+k}/\sum_{i=0}^q \beta_i^2,k=1,...,q\\
	\gamma(-k),k<0
	\end{cases},$$
	where $\beta_0 = 1$ is used.
\end{definition}


\begin{example}[MLE for invertible MA(1)]
Consider an invertible $MA(1)$ 
$$X_t = \mu + Z_t + \beta_1 Z_{t-1},$$
where $Z_t\sim WN(0, \sigma^2)$ and $\abs{\beta_1}\leq 1$.

Let $(X_1,X_2,...,X_n)$ be the observations. Then the MLE is given by
$$L(\mu, \beta_1, \sigma^2) = \frac{1}{\sqrt{(2\pi)^n \abs{\Sigma}}} \exp(-\frac{1}{2}(X-\mu)^T\Sigma^{-1}(X-\mu)),$$
where  	
$$\Sigma = \sigma^2(1 + \beta_1^2)\begin{bmatrix}
1 & \rho & 0 & \cdots & 0\\ 
\rho & 1 & \rho & \cdots & 0\\ 
0 & \rho & 1 & \cdots & 0\\ 
\vdots & \vdots & \vdots & \ddots & \vdots\\ 
0 & 0 & 0 & \cdots & 1
\end{bmatrix}, \rho = \frac{\beta_1}{\sigma^2(1 + \beta_1^2)}.$$	
\end{example}


\begin{remark}[model checking]
	Let $x_t$ be the time series observation and $\hat{x}_t$ be the linear optimal prediction. Let $r_t = x_t - \hat{x}_t$ be the residual. 	
	We can use the serial correlation(\autoref{ch:time-series-analysis:def:LjungBoxSerialCorrelationTest}) to check whether the residual is correlated or not. If the model we use is proper, then there should be no correlation among the residuals.
\end{remark}

\subsubsection{Forecasting}

\begin{definition}[best least square forecast]\cite[54]{tsay2005analysis}
Consider a time series $\{X_t\}$. Suppose we are currently at time index $h$. Let $\hat{X}_{h+l}$ be the forecast of $X_{t+h}$, where the positive integer $l$ is the forecast horizon. Let $\cF_t$ be a $\sigma$ algebra representing all the information upto $t$. 
We say $\hat{X}_{h+l}$ is the \textbf{best least square forecast} of $X_{h+l}$ if
$$E[(X_{h+l} - \hat{X}_{h+l})^2|\cF_h] = \min_{g} E[(X_{h+l} - g)^2|\cF_h], $$
where $g$ is all the random variables measurable with respect to $\cF_h$.   	
\end{definition}


\begin{remark}[interpretation]
Note that the forecast $\hat{X}_{h+l}$ is random variable instead of a single number. 
\end{remark}

\begin{lemma}[best least square forecasting for MA(1) with observed shocks]\cite[62]{tsay2005analysis}
Consider an MA(1) process given by $$X_t = Z_t + \theta_1 Z_{t-1} + c_0, Z_t\sim WN(0,\sigma^2).$$
Let $\cF_t$ denote the information available upto $t$. It follows that
\begin{itemize}
	\item the 1-step ahead best least-square forecast is
	$$\hat{X}_{t+1}\triangleq E[X_{t+1}|\cF_t] = c_0 + \theta_1Z_t,$$
	and the error is $E[(X_{t+1} - \hat{X}_{t+1})^2|\cF_t] = \sigma^2.$	
	\item the 2-step ahead best least-square forecast is
	$$\hat{X}_{t+2} \triangleq E[X_{t+2}|\cF_t]= c_0,$$
	and the error is $E[(X_{t+2} - \hat{X}_{t+2})^2|\cF_t] = (1 + \theta_1^2)\sigma^2.$	
	
	\item the m-step($m > 2$) ahead best least-square forecast is
	$$\hat{X}_{t+m} \triangleq E[X_{t+m}|\cF_t] = c_0,$$
		and the error is $E[(X_{t+m} - \hat{X}_{t+m})^2|\cF_t] = (1 + \theta_1^2)\sigma^2.$
\end{itemize}
\end{lemma}
\begin{proof}
We use the least-square minimizing property of conditional expectation(\autoref{ch:theory-of-probability:th:leastSquareMinimizingConditionalExpectation}). We have	
(1) 
\begin{align*}
\hat{X}_{t+1} &= E[Z_{t+1} + \theta_1 Z_{t} + c_0|\cF_t] \\
			 &= E[Z_{t+1}|\cF_t] + E[\theta_1 Z_{t}|\cF_t] + E[c_0|\cF_t] \\
			 &=0 + \theta_1 Z_{t} + c_0 \\
			 &=\theta_1 Z_{t} + c_0
\end{align*}
where we use the property of taking-out-known of conditional expectation(\autoref{ch:theory-of-probability:th:conditionalexpectationproperty}).
To get the error, we have
$$E[(X_{t+1} - \hat{X}_{t+1})^2|\cF_t] = E[Z_{t+1}^2|\cF_t] = \sigma^2.$$
(2) 
\begin{align*}
\hat{X}_{t+2} &= E[Z_{t+2} + \theta_1 Z_{t+1} + c_0|\cF_t] \\
&= E[Z_{t+2}|\cF_t] + E[\theta_1 Z_{t+1}|\cF_t] + E[c_0|\cF_t] \\
&=0 + 0 + c_0 \\
&= c_0
\end{align*}
where we use the property of taking-out-known of conditional expectation(\autoref{ch:theory-of-probability:th:conditionalexpectationproperty}).
To get the error, we have
$$E[(X_{t+2} - \hat{X}_{t+2})^2|\cF_t] = E[(Z_{t+2} + \theta_1Z_{t+1})^2|\cF_t] = (1 + \theta_1^2)\sigma^2.$$
(3) same as (1)(2). 
To get the error, we have
$$E[(X_{t+m} - \hat{X}_{t+m})^2|\cF_t] = E[Z_{t+m}|\cF_t] = E[(Z_{t+m} + \theta_1 Z_{t+m-1})^2|\cF_t] = (1 + \theta_1^2)\sigma^2.$$
\end{proof}


\begin{lemma}[best least square forecasting for MA(q) with observed shocks]\cite[62]{tsay2005analysis}
	Consider an MA(q) process given by $$X_t = Z_t + \theta_1 Z_{t-1} + \theta_2 Z_{t-2} + ... + \theta_q Z_{t-q} + c_0, Z_t\sim WN(0,\sigma^2).$$ 
	Let $\cF_t$ denote the information available upto $t$. It follows that
	\begin{itemize}
		\item the 1-step ahead best least-square forecast is
		$$\hat{X}_{t+1} \triangleq E[X_{t+1}|\cF_t] = c_0 + \theta_1Z_t + ... + \theta_q Z_{t-q+1},$$
		and the error is $E[(X_{t+1} - \hat{X}_{t+1})^2|\cF_t] = \sigma^2.$	
		\item the 2-step ahead best least-square forecast is
		$$\hat{X}_{t+2} \triangleq E[X_{t+2}|\cF_t] = \theta_2 Z_{t} + ... + \theta_q Z_{t-q+2} + c_0,$$
		and the error is $E[(X_{t+2} - \hat{X}_{t+2})^2|\cF_t] = (1 + \theta_1^2)\sigma^2.$	
		\item the m-step($3\leq m \leq q$) ahead best least-square forecast is
		$$\hat{X}_{t+m} \triangleq E[X_{t+m}|\cF_t] = c_0 + \sum_{i=m}^q \theta_i Z_{t-i+m},$$
		and the error is $E[(X_{t+m} - \hat{X}_{t+m})^2|\cF_t] = \sigma^2(\sum_{i=1}^{m-1} \theta_i^2),\theta_0 = 1.$
		
		\item the m-step($m > q$) ahead best least-square forecast is
		$$\hat{X}_{t+m}\triangleq E[X_{t+m}|\cF_t] = c_0,$$
		and the error is  $E[(X_{t+m} - \hat{X}_{t+m})^2|\cF_t] = \sigma^2(\sum_{i=1}^{q} \theta_i^2),\theta_0 = 1.$
	\end{itemize}
\end{lemma}
\begin{proof}
	We use the least-square minimizing property of conditional expectation(\autoref{ch:theory-of-probability:th:leastSquareMinimizingConditionalExpectation}). We have	
	(1) 
	\begin{align*}
	\hat{X}_{t+1} &= E[Z_{t+1} + \theta_1 Z_{t} + ... + \theta_q Z_{t-q+1} + c_0|\cF_t] \\
	&= E[Z_{t+1}|\cF_t] + E[\theta_1 Z_{t}|\cF_t] + E[\theta_2 Z_{t-1} + ... +\theta_q Z_{t-q+1} + c_0|\cF_t] \\
	&=0 + \theta_1 Z_{t} + ... + \theta_q Z_{t-q+1} + c_0 \\
	&=\theta_1 Z_{t} + ... + \theta_q Z_{t-q+1} + c_0
	\end{align*}
	where we use the property of taking-out-known of conditional expectation(\autoref{ch:theory-of-probability:th:conditionalexpectationproperty}).
	To get the error, we have
	$$E[(X_{t+1} - \hat{X}_{t+1})^2|\cF_t] = E[Z_{t+1}^2|\cF_t] = \sigma^2.$$
	(2) 
	\begin{align*}
\hat{X}_{t+2} &= E[Z_{t+2} + \theta_1 Z_{t+1} + ... + \theta_q Z_{t-q+2} + c_0|\cF_t] \\
&= E[Z_{t+2} + \theta_1 Z_{t+1}|\cF_t] + E[\theta_2 Z_{t}|\cF_t] + E[(\theta_3 Z_{t-1} + ... +\theta_q Z_{t-q+2} + c_0)^2|\cF_t] \\
&=0 + \theta_2 Z_{t} + ... + \theta_q Z_{t-q+2} + c_0 \\
&=\theta_2 Z_{t} + ... + \theta_q Z_{t-q+2} + c_0
\end{align*}
	where we use the property of taking-out-known of conditional expectation(\autoref{ch:theory-of-probability:th:conditionalexpectationproperty}).
	To get the error, we have
	$$E[(X_{t+1} - \hat{X}_{t+1})^2|\cF_t] = E[Z_{t+2} + \theta_1Z_{t+1}|\cF_t] = (1 + \theta_1^2)\sigma^2.$$
	(3)
	(4) same as (1)(2).
\end{proof}

\begin{lemma}[best least square forecasting for MA(q) without observed shocks]\cite[62]{tsay2005analysis}
	Consider an MA(q) process given by $$X_t = Z_t + \theta_1 Z_{t-1} + \theta_2 Z_{t-2} + ... + \theta_q Z_{t-q} + c_0, Z_t\sim WN(0,\sigma^2).$$ 
	Let $\cF_t$ denote the information available upto $t$ but not including the information about the shocks It follows that
	\begin{itemize}
		\item the 1-step ahead best least-square forecast is
$$\hat{X}_{t+1} \triangleq E[X_{t+1}|\cF_t] = c_0,$$
and the error is $E[(X_{t+1} - \hat{X}_{t+1})^2|\cF_t] = \sigma^2.$	
\item the m-step($m \geq 1$) ahead best least-square forecast is
$$\hat{X}_{t+m}\triangleq E[X_{t+m}|\cF_t] = c_0,$$
and the error is  $E[(X_{t+m} - \hat{X}_{t+m})^2|\cF_t] = \sigma^2(\sum_{i=1}^{m} \theta_i^2),\theta_0 = 1.$
	\end{itemize}
\end{lemma}
\begin{proof}
	We use the least-square minimizing property of conditional expectation(\autoref{ch:theory-of-probability:th:leastSquareMinimizingConditionalExpectation}). We have	
	(1) 
	\begin{align*}
	\hat{X}_{t+1} &= E[Z_{t+1} + \theta_1 Z_{t} + ... + \theta_q Z_{t-q+1} + c_0|\cF_t] \\
	&= E[Z_{t+1}|\cF_t] + E[\theta_1 Z_{t}|\cF_t] + E[\theta_2 Z_{t-1} + ... +\theta_q Z_{t-q+1} + c_0|\cF_t] \\
	&=0 + 0 + ... + 0 + c_0 
	\end{align*}
	To get the error, we have
	$$E[(X_{t+1} - \hat{X}_{t+1})^2|\cF_t] = E[(Z_{t+1} + \theta_1 Z_{t} + ... + \theta_q Z_{t-q+1})^2|\cF_t] = \sigma^2(\sum_{i=1}^{m} \theta_i^2),\theta_0 = 1.$$

	(2) same as (1).
\end{proof}

\begin{remark}[practical consideration of observability of shocks]
In reality, we cannot observe shocks; we can only observe $X_t$.	
\end{remark}

\subsection{AR process}
\subsubsection{Basic properties}
\begin{definition}[autoregressive operator of order $q$]
\cite[11]{box2015time}The autoregressive operator of order $q$ is defined as
$$\phi_q(B) = 1 - \phi_1 B - \phi_2 B^2 - ... - \phi_q B^q$$
where $\phi_i$ are constants.
\end{definition}

\begin{definition}[AR process of order $p$]\cite[36]{chatfield2003analysis}\index{AR process}\index{autoregressive process}\index{AR(1) process}
\begin{itemize}
	\item A process $\{X_t\}$ is called an autoregressive(AR) process of order $p$ if
	$$X_t = a_1 X_{t-1} + a_2 X_{t-2} + ... + a_p X_{t-p} + Z_t, Z_t\sim WN(0,\sigma^2)$$
	or
	$$\theta_p(B)X_t = Z_t.$$
	\item Specifically, 
	$AR(1)$ process has the following form
	$$X_t = a_1X_{t-1} + Z_t.$$
\end{itemize}	
\end{definition}



\begin{lemma}[$AR(1)$ properties]\label{ch:time-series-analysis:th:AR(1)processBasicProperties}
The first order autoregressive process $AR(1)$, given by,
	$$X_t = a_1X_{t-1} + Z_t, Z_t\sim WN(0,\sigma^2), X_0=x_0, \abs{a}<1,$$
 has the following property:
\begin{itemize}
    \item It has solution
    $$X_t = a^t X_0 + \sum_{i=1}^{t} a^{i-1}Z_{t-i+1}  $$
    \item Mean and variance
    \begin{align*}
    E[X_t] &= 0 \\
    Var[X_t] &= \sigma^2(1+a^2+a^4 + \cdots + a^{2(t-1)}) = \sigma^2\frac{1-a^{2t}}{1-a^2} \\ 
    Var[X_t] &\to \sigma^2/(1-a^2), as~t\to\infty
    \end{align*}
    \item covariance
    $$Cov[X_tX_s] = a^{t-s}Var[X_s], t\geq s$$.
    \item $$\gamma(k) = \triangleq  = \begin{cases} a^k Var(X_t), k\geq 0\\
        \gamma(-k), k<0
    \end{cases}$$
    \item $$\rho(k) = \begin{cases} a^k , k\geq 0\\
    \rho(-k), k<0
    \end{cases}$$
\end{itemize}
\end{lemma}
\begin{proof}
(1)
\begin{align*}
X_t &= aX_{t-1}+Z_t \\
&= a(aX_{t-2}+Z_{t-1})+Z_t\\
&= a^2 X_{t-2} + a Z_{t-1} + Z_t\\
&= a^3 X_{t-3} + a^2Z_{t-2} + a Z_{t-1} + Z_t\\
&= a^t X_0 + \sum_{i=1}^{t} a^{i-1}Z_{t-i+1} 
\end{align*}

(2) Directly from (1). Use the result of variance of sum of normal random variables; (3)(4)(5)
\begin{align*}
&Cov[X_tX_s] \\
=& Cov[(a^{t-s}X_s + \sum_{j=1}^{t-s} a^{t-s-j}Z_{s+j}),X_s] \\
=& Cov[a^{t-s}X_s,X_s] \\
=& a^{t-s}Var[X_s] \\
\end{align*}
\end{proof}


\begin{remark}[stationarity]
Only when $\abs{a}<1$, such process can be stationary, otherwise will blow up.
\end{remark}

\begin{remark}[connection with Ornstein-Uhlenbeck process]
Note that a stationary AR(1) process can be viewed as a discrete-time version of OU processes(\autoref{ch:theory-of-stochastic-process:sec:OUProcesses}).
\end{remark}


\subsubsection{stationarity and invertibility condition}
\begin{lemma}[stationarity condition for AR(1)]
\cite[54]{box2015time}The AR(1) process $X_t = \phi_1 X_{t-1} + Z_t$ has the necessary condition of stationarity of the roots of $1-\phi_1 t = 0$ has roots lying outside the unit circle, i.e., $\abs{t^*} > 1$; or equivalently, $\abs{\phi_1} <1$.
\end{lemma}
\begin{proof}
In lag polynomial form, we have $$(1-\phi_1 B)X_t = Z_t \Rightarrow X_t = (1-\phi_1 B)^{-1}Z_t = \sum_{i=0}^\infty \phi^i_1 B^i Z_t$$
where we know that $\abs{\phi_1}<1$ is necessary for the variance to be finite, which is equivalent to the condition of $\abs{t}=\abs{\phi_1^{-1}} > 1$  from $\abs{t\phi_1}=1$, where $t$ is the root of $1-\phi_1 t = 0$. 
\end{proof}


\begin{remark}
One example for $AR(1)$ to be nonstationary is the random walking process $X_t = X_{t-1} + W_t$; Even though the mean is 0, but its variance,i.e., $\gamma(0)$ is changing with time.
\end{remark}


\begin{lemma}[condition of stationarity]\label{ch:time-series-analysis:th:ARstationarycondition}
\cite[54]{box2015time} \hfill
\begin{itemize}
	\item The $AR(q)$ process $X_t = \phi_1 X_{t-1} + \phi_2 X_{t-2} + ... + \phi_q X_{t-q} + Z_t$ is stationary of the roots of polynomial $\phi(t) = 1-\phi_1 t - \phi_2 t^2 - \phi_3 t^3 ... = 0$ must all lying outside the unit circle,i.e., $\abs{t^*} > 1$.
	\item Another necessary condition for stationary is
	$$\abs{\phi_i}\leq 1, \forall i=1,2,...,q.$$
\end{itemize}
\end{lemma}
\begin{proof}
(1)	
From \autoref{ch:time-series-analysis:th:inverselagpolynomial}. The condition ensures that AR(q) can be represented as a MA process, and further MA process are stationary(\autoref{ch:time-series-analysis:th:MAprocessisstationary}).
(2) 
\end{proof}
 


\begin{lemma}[AR process is always invertible]
\cite[57]{box2015time}An AR(q) process $X_t = \phi_1 X_{t-1} + \phi_2 X_{t-2} + ... + \phi_q X_{t-q} + Z_t$ is alway intertible.
\end{lemma}
\begin{proof}
From the definition of invertible process(\autoref{ch:time-series-analysis:def:invertibility}).
\end{proof}


\subsubsection{Yule-Walker equation}

\begin{theorem}[Yule-Walker equation for AR(q) process]\label{ch:time-series-analysis:YuleWalderEquationForAR(q)}
Given an zero-mean $AR(p)$ model,
$$X_t = a_1 X_{t-1} + a_2 X_{t-2} + ... + a_p X_{t-p} + Z_t, Z_t\sim WN(0,\sigma^2).$$
The autocovariance and the coefficients $a_1,a_2,...,a_q$ are related in the following ways:
\begin{itemize}
	\item $$\gamma(h) = a_1\gamma(h-1) + a_2\gamma(h-2) + \cdots + a_q \gamma(h-q), ~for~ h=1,...,q;$$
	or in matrix form
	$$
	\begin{bmatrix}
	\gamma(1)\\
	\gamma(2)\\
	\vdots\\
	\gamma(q)
	\end{bmatrix} = \underbrace{\begin{bmatrix}
	\gamma(0) & \gamma(1) & \cdots & \gamma(q-1)\\ 
	\gamma(1) & \gamma(0) & \cdots & \gamma(q-2)\\ 
	\vdots & \vdots & \ddots & \cdots\\ 
	\gamma(q-1) & \gamma(q-2) & \cdots & \gamma(0) 
	\end{bmatrix}}_\Gamma \begin{bmatrix}
	a_1\\
	a_2\\
	\vdots\\
	a_q
	\end{bmatrix}.$$
	In particular, the matrix $\Gamma$ is symmetric positive definite when $X_t$s are not perfectly correlated. 
	\item If both sides divided by $\gamma(0)$, we have
	$$
	\begin{bmatrix}
	\rho(1)\\
	\rho(2)\\
	\vdots\\
	\rho(q)
	\end{bmatrix} = \begin{bmatrix}
	1 & \rho(1) & \cdots & \rho(q-1)\\ 
	\rho(1) & 1 & \cdots & \rho(q-2)\\ 
	\vdots & \vdots & \ddots & \cdots\\ 
	\rho(q-1) & \rho(q-2) & \cdots & 1 
	\end{bmatrix} \begin{bmatrix}
	a_1\\
	a_2\\
	\vdots\\
	a_q
	\end{bmatrix}.$$
	\item $$\gamma(0) = a_1\gamma(1) + a_2\gamma(2) + \cdots + a_p\gamma(q) + \sigma^2.$$
	\item 
\end{itemize} 	
\end{theorem}
\begin{proof}
(1)(2)Given the zero mean $AR(q)$ model,
$$X_t = a_1 X_{t-1} + a_2 X_{t-2} + ... + a_p X_{t-p} + Z_t,(*)$$	
we can multiply both sides by $X_{t-h}$ for $h=1,...,p$, and get
$$X_tX_{t-h} = a_1 X_{t-1}X_{t-h} + a_2 X_{t-2}X_{t-h} + ... + a_p X_{t-p}X_{t-h} + Z_tX_{t-h}.$$
Take expectation and we get
$$\gamma(h) = a_1\gamma(h-1) + a_2\gamma(h-2) + \cdots + a_q \gamma(h-q).$$

The positive definiteness of $\Gamma$ can be seen from the fact that $$\Gamma = E[XX^T],X = (X_1,X_2,...,X_q),$$
and for any nonzero $p\in \R^q$, $q^T\Gamma q = Var[q^TX] > 0. $
(3)If we take $h = 0$, we will get
$$X_tX_{t} = a_1 X_{t-1}X_{t} + a_2 X_{t-2}X_{t} + ... + a_p X_{t-p}X_{t} + Z_tX_{t}.$$
Take expectation, we get
$$\gamma(0) = a_1\gamma(1) + a_2\gamma(2) + \cdots + a_p\gamma(q) + \sigma^2.$$
\end{proof}

\begin{example}
Consider the AR(2) process given by
$$X_n = Z_n + a_1 X_{n-1} + a_2 X_{n-2}.$$
The Yule-Walker equations are 
$$\begin{bmatrix}
1 ~ \rho_1 \\
\rho_1 ~ 1
\end{bmatrix}\begin{bmatrix}
a_1 \\
a_2
\end{bmatrix} = \begin{bmatrix}
\rho_1 \\
\rho_2
\end{bmatrix}.$$

Solving the Yule-Walker equation, we obtain
$$\rho_1 = \frac{a_1}{1 - a_2}, \rho_2 = \frac{a_1^2}{1 - a_2} + a_2.$$	
\end{example}

\subsubsection{Identification and parameter estimation}

\begin{remark}[order identification]
We can use the PACF(\autoref{ch:time-series-analysis:sec:PartialAutocorrelationFunction}) to identify the order of AR process.	
\end{remark}

\begin{theorem}[method of moments parameter estimation via Yule-Walker equation]
Given an zero-mean $AR(p)$ model,
$$X_t = a_1 X_{t-1} + a_2 X_{t-2} + ... + a_p X_{t-p} + Z_t, Z_t\sim WN(0,\sigma^2).$$
The autocovariance and the coefficients $a_1,a_2,...,a_q$ are related in the following ways:
\begin{itemize}
	\item Let $\hat{\rho} = (\hat{\rho}(1),\hat{\rho}(2),...,\hat{\rho}(q))^T$, $a = (a_1,a_2,...,a_q)$, and $\hat{\Gamma}$ be the matrix of
	$$\begin{bmatrix}
	1 & \hat{\rho}(1) & \cdots & \hat{\rho}(q-1)\\ 
	\hat{\rho}(1) & 1 & \cdots & \hat{\rho}(q-2)\\ 
	\vdots & \vdots & \ddots & \cdots\\ 
	\hat{\rho}(q-1) & \hat{\rho}(q-2) & \cdots & 1 
	\end{bmatrix},$$
	then $\hat{a} = (\hat{\Gamma})^{-1}\hat{\rho}.$
	\item 
	$$\hat{\sigma}^2 = \hat{\gamma}(0) - \hat{a}_1\hat{\gamma}(1) - \hat{a}_2\hat{\gamma}(2) - \cdots \hat{a}_q\hat{\gamma}(q)$$
\end{itemize}
\end{theorem}



\begin{theorem}[least square estimation for AR(q)]\label{ch:time-series-analysis:th:LeastSquareParameterEstimationAR(q)}
	Consider the AR(q) process given by
	
	\begin{itemize}
		\item The correlation parameter $\beta = (\beta_1,\beta_2,..,\beta_q)^T$ 
		$$Y = \begin{bmatrix}
		x_{q+1}\\
		x_{q+2}\\
		\vdots \\
		x_N
		\end{bmatrix}, X = \begin{bmatrix}
		x_q & x_{q-1} & \cdots & x_1 \\
		x_{q+1} & x_q & \cdots & x_2\\
		\vdots & \vdots & \ddots & \vdots\\
		x_{N-1} & x_{N-2} & \cdots & x_{N-q}
		\end{bmatrix},$$
		and the $Y = X\beta $ model gives the least square estimator 
		$$\hat{\beta} = (X^TX)^{-1}X^TY.$$
		\item The estimation of $\sigma$ is given by
		$$\hat{\sigma}^2 = \frac{SSE}{n-p-1}.$$		
	\end{itemize}	
\end{theorem}
\begin{proof}
See \autoref{ch:statistical-models:th:leastSquareSolution}	
\end{proof}

\begin{remark}[connection of least square approach to conditional likelihood approach]
	\begin{itemize}
		\item Note that we can decompose a full distribution as the product of conditional distributions given by 
		$$f(X_N,X_{N-1},...,X_1) = f(X_N|X_{N-1},...,X_{N-q})f(X_{N-1}|X_{N-2},...,X_{N-q-1})\cdots f(X_{q+1}|X_{q},...,X_{1}).$$
		If the conditional distribution is Gaussian, then the maximum likelihood will give the same formula as the least square solution.
		\item In contrast, full distribution likelihood approach requires the auto-covariance structure of an $AR(q)$ process, which is difficult to obtain. 
	\end{itemize}
\end{remark}


\begin{lemma}[least square estimation of the correlation for AR(1)]\cite[53]{chatfield2003analysis}\cite[46]{tsay2005analysis}\label{ch:time-series-analysis:th:leastSquareEstimationOfCorrelationAR(1)}
Consider the AR(1) process given by
\begin{itemize}
	\item The correlation parameter $\rho$ as a solution to	
	$$\min_{\rho} \sum_{i=2}^N (x_i - \rho x_{i-1})^2$$
	is given by
	$$\hat{\rho} = \frac{\sum_{t=1}^{N-1} x_t x_{t+1}}{\sum_{t=1}^{N-1} x_t^2}.$$	
	Alternatively, 
	$$Y = \begin{bmatrix}
	x_2\\
	x_3\\
	\vdots\\
	x_N
	\end{bmatrix}, X = \begin{bmatrix}
	x_1\\
	x_2\\
	\vdots\\
	x_{N-1}
	\end{bmatrix},$$
	and the $Y = X\rho $ model gives the least square estimator 
	$$\hat{\rho} = (X^TX)^{-1}X^TY.$$
	\item The estimation of $\sigma$ is given by
\end{itemize}
\end{lemma}
\begin{proof}
Use the least square solution(\autoref{ch:statistical-models:th:leastSquareSolution}).	
\end{proof}


\begin{example}[least square estimation for AR(2)]
	Consider the AR(2) process given by
	
	\begin{itemize}
		\item The correlation parameter $\beta = (\beta_1,\beta_2)^T$ 
		$$Y = \begin{bmatrix}
		x_3\\
		x_4\\
		\vdots\\
		x_N
		\end{bmatrix}, X = \begin{bmatrix}
		x_2 & x_1\\
		x_3 & x_2\\
		\vdots & \vdots\\
		x_{N-1} & x_{N-2}
		\end{bmatrix},$$
		and the $Y = X\beta $ model gives the least square estimator 
		$$\hat{\beta} = (X^TX)^{-1}X^TY.$$
		\item The estimation of $\sigma$ is given by
		
	\end{itemize}	
\end{example}






\begin{remark}[model checking]
Let $x_t$ be the time series observation and $\hat{x}_t$ be the linear optimal prediction. Let $r_t = x_t - \hat{x}_t$ be the residual. 	
	We can use the serial correlation(\autoref{ch:time-series-analysis:def:LjungBoxSerialCorrelationTest}) to check whether the residual is correlated or not. If the model we use is proper, then there should be no correlation among the residuals.
\end{remark}

\subsubsection{Forecasting}
\begin{lemma}[best least square forecasting for AR(1)]\cite[55]{tsay2005analysis}
	Consider an AR(1) process given by $$X_t = a_1 X_{t-1} + Z_t + c_0, Z_t\sim WN(0,\sigma^2).$$ Let $\cF_t$ denote the information available upto $t$. It follows that
	\begin{itemize}
		\item the 1-step ahead best least-square forecast is
		$$\hat{X}_{t+1} \triangleq E[X_{t+1}|\cF_t]= c_0 + a_1X_t,$$
		and the error is $E[(X_{t+1} - \hat{X}_{t+1})^2|\cF_t] = \sigma^2.$	
		\item the 2-step ahead best least-square forecast is
		$$\hat{X}_{t+2}\triangleq E[X_{t+2}|\cF_t] = c_0 + a_1\hat{X}_{t+1},$$
		and the error is $E[(X_{t+2} - \hat{X}_{t+2})^2|\cF_t] = (1 + a_1^2)\sigma^2.$	
		
		\item the m-step($m > 2$) ahead best least-square forecast is
		$$\hat{X}_{t+m}\triangleq E[X_{t+m}|\cF_t] = c_0 + a_1 \hat{X}_{t+m-1},$$
		and the error is $$E[(X_{t+m} - \hat{X}_{t+m})^2|\cF_t] = \sigma^2 + a_1^2E[(X_{t+m-1} - \hat{X}_{t+m-1})^2|\cF_t] = \sigma^2 \frac{a_1^m - 1}{a_1 - 1} .$$
	\end{itemize}
\end{lemma}
\begin{proof}
	We use the least-square minimizing property of conditional expectation(\autoref{ch:theory-of-probability:th:leastSquareMinimizingConditionalExpectation}). We have	
	(1) 
	\begin{align*}
	\hat{X}_{t+1} &= E[Z_{t+1} + a_1 X_{t} + c_0|\cF_t] \\
	&= E[Z_{t+1}|\cF_t] + E[a_1 X_{t}|\cF_t] + E[c_0|\cF_t] \\
	&=0 + a_1 X_{t} + c_0 \\
	&=a_1 X_{t} + c_0
	\end{align*}
	where we use the property of taking-out-known of conditional expectation(\autoref{ch:theory-of-probability:th:conditionalexpectationproperty}).
	To get the error, we have
	$$E[(X_{t+1} - \hat{X}_{t+1})^2|\cF_t] = E[Z_{t+1}^2|\cF_t] = \sigma^2.$$
	(2) 
	\begin{align*}
	\hat{X}_{t+2} &= E[Z_{t+2} + a_1 X_{t+1} + c_0|\cF_t] \\
	&= E[Z_{t+2}|\cF_t] + E[a_1 X_{t+1}|\cF_t] + E[c_0|\cF_t] \\
	&= 0 + a_1\hat{X}_{t+1} + E[c_0|\cF_t] \\
	&=0 + a_1(a_1X_t + c_0) + c_0 \\
	&= a_1^2 X_t + a_1c_0 + c_0
	\end{align*}
	where we use the property of taking-out-known of conditional expectation(\autoref{ch:theory-of-probability:th:conditionalexpectationproperty}).
	To get the error, we have
	$$E[(X_{t+2} - \hat{X}_{t+2})^2|\cF_t] = E[(Z_{t+2} + a_1(X_{t+1} - \hat{X}_{t+1}))^2|\cF_t] = (1 + a^2)\sigma^2.$$
	(3) same as (1)(2). 
	To get the error, we have
	$$E[(X_{t+m} - \hat{X}_{t+m})^2|\cF_t] = E[(Z_{t+m} + a_1(X_{t+m-1} - E[X_{t+m-1}|\cF_t])^2|\cF_t] =\sigma^2 + a_1^2E[(X_{t+m-1} - \hat{X}_{t+m-1})^2|\cF_t] .$$
	Continue the recursive relation and we will get the result.
\end{proof}

\begin{lemma}[best least square forecasting for AR(q)]\cite[55]{tsay2005analysis}
	Consider an AR(q) process given by $$X_t = a_1 X_{t-1} + ... + a_q X_{t-q} + Z_t + c_0, Z_t\sim WN(0,\sigma^2).$$ Let $\cF_t$ denote the information available upto $t$. It follows that
	\begin{itemize}
		\item the 1-step ahead best least-square forecast is
		$$\hat{X}_{t+1} \triangleq E[X_{t+1}|\cF_t] = c_0 + a_1X_t + ... + a_q X_{t-q+1},$$
		and the error is $$E[(X_{t+1} - \hat{X}_{t+1})^2|\cF_t] =E[Z_{t+1}^2|\cF_t] = \sigma^2.$$	
		\item the 2-step ahead best least-square forecast is
		$$\hat{X}_{t+2} \triangleq E[X_{t+2}|\cF_t] = c_0 + a_1\hat{X}_{t+1} + a_2 X_t + ... +a_q X_{t-q+2},$$
		and the error is $E[(X_{t+2} - \hat{X}_{t+2})^2|\cF_t] = (1 + a_1^2)\sigma^2.$	
		\item the 3-step ahead best least-square forecast is
		$$\hat{X}_{t+3}\triangleq E[X_{t+3}|\cF_t] = c_0 + a_1\hat{X}_{t+2} + a_2\hat{X}_{t+1} + a_3X_{t} + ... + a_q X_{t-q+3},$$
		and the error is $E[(X_{t+3} - \hat{X}_{t+3})^2|\cF_t] = (1 + a_1^2 + (a_1^2 + a_2)^2)\sigma^2.$
		\item the m-step($m > q$) ahead best least-square forecast is
		$$\hat{X}_{t+m}\triangleq E[X_{t+m}|\cF_t] = c_0 + a_1 \hat{X}_{t+m-1}+...+a_q \hat{X}_{t+m-q}.$$

	\end{itemize}
\end{lemma}
\begin{proof}
	We use the least-square minimizing property of conditional expectation(\autoref{ch:theory-of-probability:th:leastSquareMinimizingConditionalExpectation}). We have	
	(1) 
	\begin{align*}
	\hat{X}_{t+1} &= E[Z_{t+1} + a_1 X_{t} + c_0|\cF_t] \\
	&= E[Z_{t+1}|\cF_t] + E[a_1 X_{t}|\cF_t] + E[c_0|\cF_t] \\
	&=0 + a_1 X_{t} + c_0 \\
	&=a_1 X_{t} + c_0
	\end{align*}
	where we use the property of taking-out-known of conditional expectation(\autoref{ch:theory-of-probability:th:conditionalexpectationproperty}).
	To get the error, we have
	$$E[(X_{t+1} - \hat{X}_{t+1})^2|\cF_t] = E[Z_{t+1}^2|\cF_t] = \sigma^2.$$
	(2) 
	\begin{align*}
	\hat{X}_{t+2} &= E[Z_{t+2} + a_1 X_{t+1} + c_0|\cF_t] \\
	&= E[Z_{t+2}|\cF_t] + E[a_1 X_{t+1}|\cF_t] + E[c_0|\cF_t] \\
	&= 0 + a_1\hat{X}_{t+1} + E[c_0|\cF_t] \\
	&=0 + a_1(a_1X_t + c_0) + c_0 \\
	&= a_1^2 X_t + a_1c_0 + c_0
	\end{align*}
	where we use the property of taking-out-known of conditional expectation(\autoref{ch:theory-of-probability:th:conditionalexpectationproperty}).
	To get the error, we have
	$$E[(X_{t+2} - \hat{X}_{t+2})^2|\cF_t] = E[(Z_{t+2} + a_1(X_{t+1} - \hat{X}_{t+1}))^2|\cF_t] = (1 + a^2_1)\sigma^2.$$
	(3)(4) same as (1)(2). 
	To get the error, we have
\begin{align*}
E[(X_{t+3} - \hat{X}_{t+3})^2|\cF_t] &= E[(Z_{t+3} + a_1(X_{t+2} - \hat{X}_{t+2}) + a_2 (X_{t+1} - \hat{X}_{t+1}))^2|\cF_t] \\
&= E[(Z_{t+3} + a_1 Z_{t+2} + a_1(a_1(X_{t+1} - \hat{X}_{t+1})) + a_2 (X_{t+1} - \hat{X}_{t+1}))^2|\cF_t] \\
&= (1 + a_1^2 + (a_1^2 + a_2)^2 )\sigma^2 
\end{align*}	
\end{proof}

\subsection{ARMA process}
\subsubsection{Basic properties}
\begin{definition}[ARMA process]
An $ARMA(p,q)$ model can be represented as
$$\theta_p(L) X_t = \phi_q(L)W_t$$
where $\theta_p$ is the AR operator of order $p$,$\phi_q$ is the MA operator of order $q$.
\end{definition}


\begin{remark}[Parameter redundancy in ARMA]
When $\theta$ and $\phi$ share a common factor, the model can be simplified. For example, $(1-0.5L)(1-1/3L)X_t = (1-0.5L)W_t \Rightarrow (1-1/3B)X_t = W_t$.
\end{remark}

\begin{lemma}[stationarity of ARMA process]\cite[95]{shumway2010time}
The $ARMA(p,q)$ process is stationary if the roots of $\theta_q(z) = 0$ satisfies $\abs{z}>1$.
\end{lemma}
\begin{proof}
See \autoref{ch:time-series-analysis:th:ARstationarycondition}.
\end{proof}

\begin{lemma}[invertibility of ARMA process]\cite[95]{shumway2010time}
An $ARMA$ model is \textbf{invertible}, i.e.,
$W_t = \phi_q(L)^{-1}\theta_p(L)X_t =  \pi(L)W_t = \sum_{j=0}^\infty \pi_j X_{t-j}$
if and only if $\phi_p(z)\neq 0$ for $\abs{z}<1$.
\end{lemma}
\begin{proof}
When $\phi_q$ is invertible, then we have $W_t = \phi_q^{-1}(L)\theta_p(L)X_t$. Therefore, it is invertible.
\end{proof}

\begin{lemma}[causal form, moving average representation of AR process]\cite[95]{shumway2010time}
An $ARMA$ model can be written as a \textbf{causal form}, i.e.
$X_t = \theta_p(L)^{-1}\phi_q(L)W_t = \psi(L)W_t = \sum_{j=0}^\infty \psi_j W_{t-j}$
if and only if $\theta_p(z)\neq 0$ for $\abs{z}<1$.
\end{lemma}
\begin{proof}
See \autoref{ch:time-series-analysis:th:ARstationarycondition} and \autoref{ch:time-series-analysis:th:inverselagpolynomial}.
\end{proof}


\subsubsection{Forcasting}




\section{Wold Representation theorem}
\begin{definition}[orthogonal projection ]
Let $X_t,t\in \Z$ be a covariance-stationary process. The random variable
$$T[X_{t+h}|X_{t-1},...,X_{t-N}] = a_0 + a_1X_{t-1} + ... + a_Nx_{t-N}$$
where coefficient $a_1,...,a_N$ are such that 
$$E[(X_{t+h} - T[X_{t+h}|X_{t-1},...,X_{t-N}])^2]$$ is minimun, is called \textbf{orthogonal projection} of $X_{t+h}$ on $X_{t-1},...,X_{t-N}$.
$$T[X_{t+h}|X_{t-1},X_{t-2},...] = a_0 + a_1X_{t-1} + a_Nx_{t-2} + ...$$
\end{definition}


\begin{definition}[linearly deterministic process]\index{linearly deterministic process}
A covariance-stationary process, $X_t$, is called (linearly) deterministic if 
$$T[X_{t}|X_{t-1},X_{t-2},...] = X_t.$$
\end{definition}


\begin{remark}[implications]\hfill
\begin{itemize}
\item A stationary process $X_t$ is deterministic if $X_t$ can be predicted correctly (with zero error) using the entire past $X_{t-1},X_{t-2},...$
\item For a deterministic process, the one-step linear prediction error is zero.
\item A \textbf{nonlinear dynamical system might not be linearly deterministic}. 
\end{itemize}
\end{remark}


\begin{example}
The process $X_t = A\cos(t) + Bcos(t)$ is deterministic process because $A,B$ can be determined from past history and then the future can be predicted with zero error.
\end{example}


\begin{theorem}[Wold Representation theorem, Wold decomposition theorem]\index{Wold Representation theorem}\index{Wold decomposition theorem}\label{ch:time-series-analysis:th:wolddecompositionsingleD}
Any zero-mean nondeterministic covariance stationary time series $X_t$ can be \textbf{uniquely} decomposed as
$$X_t = V_t + S_t$$
where 
\begin{itemize}
\item $V_t$ is a linearly determinstic process
\item $S_t$ is an infinitely moving average process given as
$$S_t = \phi_\infty(L)W_t, \phi_\infty(L) = I + \phi_1 L + ...$$
$$\sum_{i=1}^\infty \phi_i^2 < \infty$$
\item $W_t \sim WN(0,\sigma^2)$
\item $E[V_t W_s] = 0,\forall t,s$
\end{itemize}
\end{theorem}

\begin{definition}[purely deterministic process]\index{purely deterministic process}
zero-mean nondeterministic covariance stationary time series $X_t$ is called \textbf{purely nondeterministic process} if in this Wold decomposition, the deterministic process component is zero, i.e. $V_t = 0$. 
\end{definition}




\begin{theorem}[multivariate Wold decomposition]\index{Wold decomposition theorem}\label{ch:time-series-analysis:th:wolddecompositionmultiD}
Any $m$ dimensional convariance stationary time series $X_t$ can be decomposed as
\begin{align*}
X_t &= V_t + \eta_t + \psi_1 \eta_{t-1} + \psi_2 \eta_{t-2} + ... \\
& = V_t + \sum_{k=0}^\infty \psi_k \eta_{t-k}
\end{align*} 
where
\begin{itemize}
\item $V_t \in \R^m$ an $m$ dimensional linearly deterministic process.
\item $\eta_t \in \R^m$ is multivariate white noise process, i.e. $E[\eta_t] = 0, E[\eta_t\eta_s^T] = \Sigma \delta_{st}$
\item $cov(\eta_t,V_s) = 0,\forall t,s$
\item $\psi_k \in \R^{m\times m}$ and $\psi_0 = I_m$, $\sum_{i=0}^\infty \psi_i\psi_i^T $ converges.
\end{itemize}
\end{theorem}

\section{Spectral analysis for stationary process}
\subsection{Spectral theory}
\begin{theorem}[spectral representation of ACF]
\cite[181]{shumway2010time} If the autocovariance function, $\gamma(h),h\in \cZ$, of a stationary process satisfies 
$$\sum_{h=-\infty}^{\infty} \abs{\gamma(h)} < \infty$$
then $\gamma(h)$ has the representation, 
$$\gamma(h) = \int_{1/2}^{1/2}e^{2\pi i \omega h f(\omega) d\omega}$$
as the inverse transform of the \textbf{spectral density}, which is defined as
$$f(\omega) = \sum_{-\infty}^{\infty} \gamma(h)e^{-2\pi i \omega h}, -1/2 \leq \omega \leq 1/2$$
\end{theorem}
Proof: (1) first the convergence of series $\sum_{-\infty}^{\infty} \gamma(h)e^{-2\pi i \omega h}$ is guaranteed by the M-test. (2) It can be verified
\begin{align*}
    \gamma(h) &= \int_{1/2}^{1/2}e^{2\pi i \omega h} f(\omega) d\omega\\
        &= \int_{1/2}^{1/2}e^{2\pi i \omega h}  \sum_{-\infty}^{h'\infty} \gamma(h')e^{-2\pi i \omega h'}d\omega \\
        &=\gamma(h')\delta(h-h')=\gamma(h)
\end{align*}
where we have use the fact or (1) uniformly convergent series can be integrated term-by-term and (2) $$\sum_{h'=-\infty}^{\infty}\int_{1/2}^{1/2}e^{2\pi i \omega (h-h')}  d\omega =\delta(h-h')$$ based on orthnormality.


\begin{remark}
Here because the $\gamma$ only take value at discrete $h$, therefore its spectral density only span on a closed interval. 
\end{remark}

\begin{remark}
This is not discrete Fourier transform, which will provide the spectral reprentation of a specific trajectory. 
\end{remark}

\begin{lemma}[white noise spectrum density]
The white noise has the spectrum density given as $f(\omega) = 1$.
\end{lemma}
Proof: for white noise, we have $\gamma(h)=\delta(h)$, plug into the definition $f(\omega)$ and get the result.



\subsection{Periodogram and discrete Fourier transform}
\cite[187]{shumway2010time} Given data $x_1,x_2,...,x_n$, we define the periodogram to $$I(\omega_j) = \abs{d(\omega_j)}^2,\omega_j = j/n$$
where
$$d(\omega_j) = n^{-1/2}\sum_{t=1}^n x_t e^{-2\pi i \omega_j t}$$
where $j=0,1,2...,n-1$.


\section{Linear nonstationary process}
\subsection{ARIMA models}
\begin{definition}[integrated process]
A series $\{X_t\}$ is integrated of order $d$, denoted as $I(d)$, if $$\nabla^d X_t = W_t$$
or 
$$(1-L)^d X_t= W_t.$$
\end{definition}


\begin{definition}[ARIMA]
An $ARIMA(p,d,q)$ model can be represented as
$$\theta_p(L)(1-L)^d X_t = \phi_q(L)W_t$$
where $\theta_p$ is the AR operator of order $p$,$\phi_q$ is the MA operator of order $q$.
If $p=0$, we get model $IMA(d,q) = ARIMA(0,d,q)$; If $q=0$, we get model $ARI(p,d)=ARIMA(p,d,q)$.
\end{definition}

\begin{remark}[How to fit the data] One idea to fit the data to the model is to first assume some $d$, take the differencing on the data, then then fit a ARMA model to the differenced data.
\end{remark}


\begin{example}\cite[140]{cowpertwait2009introductory} The model $$X_t = X_{t-1}+W_t + \beta W_{t-1} \Leftrightarrow (1-B)X_t = (1+\beta B)W_t$$
is $ARIMA(0,1,1)$.
\end{example}

\begin{example}\cite[140]{cowpertwait2009introductory} The model $$X_t = (1+\alpha)X_{t-1}-\alpha X_{t-2}+W_t \Leftrightarrow (1-B)(1-\alpha B)X_t = W_t$$
is $ARIMA(1,1,0)$.
\end{example}


\begin{remark}
This $ARIMA$ model cannot handle seasonal effect, but can handle general polynomial trends. 
\end{remark}

\begin{definition}[seasonal ARIMA]
If $d$ and $D$ are non-negative integers, then $\{X_t\}$ is a seasonal $ARIMA(p,d,q)\times (P,D,Q)_s$ process with period $s$ if the differenced series $Y_t = (1-B)^d(1-B^s)^DX_t$ is a causal $ARMA$ process defined by 

$$\phi$$
\end{definition}


\begin{definition}[seasonal operator]
The \textbf{seasonal autoregressive  operator $\Phi_P(B^s)$} with order $P$ and season $s$ is defined as
$$\Phi_P(B^s) = 1- \Phi_1 B^s -- \Phi_2 B^2s ... - - \Phi_P B^Ps$$
The \textbf{seasonal moving average  operator $\Theta_P(B^s)$} with order $Q$ and season $s$ is defined as
$$\Theta_Q(B^s) = 1- \Theta_1 B^s -- \Theta_2 B^2s ... - - \Theta_Q B^Qs$$

\end{definition}

\begin{definition}[pure seasonal ARMA]
A pure seasonal $ARMA(P,Q)_s$ process is given as
$$\Phi_P(B^s)X_t = \Theta_Q(B^s)W_t$$
\end{definition}

\begin{example}[A seasonal AR(1)]
A pure $SAR(1)_{12}$ can be written as
$$X_t = \Phi X_{t-12} + W_t$$
\end{example}

\begin{definition}[mixed seasonal ARMA]
A mixed seasonal $ARMA(p,q)\times(P,Q)_s$ process is given as
$$\Phi_P(B^s)\phi_p(B)X_t = \Theta_Q(B^s)\theta_q(B)W_t$$
\end{definition}

\begin{remark}\hfill
\begin{itemize}
    \item Note that these polynomials are commute.
    \item By setting $P,Q=0$, we can recover non-seasonal $ARMA$ model.
\end{itemize}
\end{remark}

\begin{example}[A mixed seasonal model]
Consider an $ARMA(0,1)\times(1,0)_{12}$ model:
$$X_t -\theta X_{t-12} = W_t + \phi W_{t-1}$$
\end{example}


\subsection{Unit root nonstationary process}
\subsubsection{Unit root process}

\begin{definition}[AR(p) process as a unit root process]\hfill
\begin{itemize}
	\item An AR(p) process 
	$$y_t = a_1y_{t-1} + a_2y_{t-2} + ... + a_p y_{t-p} + \epsilon_t,\epsilon_t\sim WN(0,1)$$
	is a unit root process, if the characteristic polynomial 
	$$p(z) = 1-a_1z - a_2z^2 - \cdots - a_p z^p$$
	has at least one unit root(a root equal to 1) and all other roots are outside the complex unit circle. That is, if $p(z)$ has a unit root, then
	$$p(1)=1 - a_1 - a_2 - \cdots - a_p = 0.$$
	\item An AR(1) process
		$$y_t = y_{t-1} + \epsilon_t, \epsilon_t\sim WN(0,1)$$
	is a unit root process
\end{itemize}	
\end{definition}

\begin{example}
A random walk $$y_t = y_{t-1} + \epsilon_t,\epsilon_t\sim WN(0,1)$$
is a $AR(1)$ process with unit root to the characteristic polynomial $1-z = 0$. 
\end{example}

\begin{lemma}[basic properties of AR(1) unit root process]
Consider an AR(1) process given by
		$$y_t = y_{t-1} + \epsilon_t, \epsilon_t\sim WN(0,\sigma^2),$$
		where initial condition $y_0$.
It follows that
	\begin{itemize}
		\item The solution is
		$$y_t = y_0 + \sum_{j=1}^{t}\epsilon_j.$$
		\item The mean and variance is
		$$E[y_t] = y_0, Var[y_t] = \sigma^2 t$$
		\item The long-run correlation is given by
		$$corr(y_t,y_{t+h}) = \frac{E[(\sum_{i=1}^t u_i)(\sum_{i=1}^{t+h} u_i)]}{(t\sigma^2(t+h)\sigma^2)^{0.5}} = \frac{t}{\sqrt{t(t+h)}} \to 1$$
		as $t\to \infty$.
	\end{itemize}	
\end{lemma}
\begin{proof}
(1) Just repeat substitution. (2)(3) Use (1).
\end{proof}





\begin{remark}[ in random walks]
Let $y_t = y_{t-1} + z_t,z_t\sim N(0,\sigma^2)$ be a random walk. 

\end{remark}


\subsubsection{Trend stationarity vs. unit root process}

\begin{note}[deterministic trend vs. stochastic trend]
Many time series are trending. It is important to distinguish between two important cases:
\begin{itemize}
	\item A stationary process with a deterministic trend: shocks have \textbf{transitory} effects.
	\item A process with a stochastic trend or a unit root: shocks have \textbf{permanent} effects.	
\end{itemize}
\end{note}

\begin{example}[deterministic trend vs. stochastic trend]
\begin{itemize}
	\item Consider a stationary AR(1) model with a deterministic linear trend term
	$$Y_t = \theta Y_{t-1} + a_0 + a_1 t + \epsilon_t, \epsilon_t\sim WN(0,\sigma^2),$$
	where $\abs{\theta} < 1$, and $Y_0$ is an initial value.
	\begin{itemize}
		\item 	Note that the mean and variance is given by
		$$E[Y_t] = \theta^t Y_0 + \mu + \mu_1t \to \mu + \mu_1t, as~ t\to\infty,$$
		and
		$$Var[Y_t] = \frac{\sigma^2}{1-\theta},$$
		where $\mu = a_0/(1-\theta), \mu_1=a_1/(1-\theta)$.
		
		\item $Y_t$ is not a stationary process since its mean is time dependent.However, the process $Y_t - E[Y_t]$ is a stationary process, called trend-stationary.
		\item Also note that stochastic part of $Y_t$ is stationary and shocks have transitory effects. We say that the process is mean reverting to its long-run $ \mu + \mu_1t$.
	\end{itemize}
	\item 

	
\end{itemize}	
	
\end{example}



\subsubsection{Unit root test}
\begin{definition}[Dickey-Fuller $\rho$ test]\index{Dickey-Fuller $\rho$ test} \cite[574]{hayashi2000econometrics}
Consider the AR(1) model
$$y_t = \theta y_{t-1} + \epsilon_t$$ with $T+1$ observations $\{y_0,y_1,...,y_T\}$ and the hypothesis
$$H_0: \theta = 1; H_1: \theta < 1.$$
The test statistic,known as Dickey-Fuller $\rho$ statistic, is given as
$$DF = T(\hat{\theta} - 1)$$
where $\hat{\theta}$ is the least square estimation of $\theta$ from linear regression.
\end{definition}


\begin{definition}[Dickey-Fuller $t$ test]\index{Dickey-Fuller $t$ test} \cite[574]{hayashi2000econometrics}
Consider the AR(1) model
$$y_t = \theta y_{t-1} + \epsilon_t$$ with $T+1$ observations $\{y_0,y_1,...,y_T\}$ and the hypothesis
$$H_0: \theta = 1; H_1: \theta < 1.$$
The test statistic,known as Dickey-Fuller $t$ statistic, is given as
$$DF_t = \frac{\hat{\theta} - 1}{SSE[\hat{\theta}/(n-2)]}$$
where $\hat{\theta}$ is the least square estimation of $\theta$ from linear regression, such that
$$\hat{\theta} = \frac{\sum_{i=1}^T y_{i-1}y_i}{\sum_{i=1}^T y_i^2}$$
\end{definition}

\begin{definition}[augmented Dickey-Fuller test]\index{augmented Dickey–Fuller test} 

\end{definition}



\section{Autoregressive conditional heteroscedastic model}
\subsection{ARCH models}
\subsubsection{The motivation}

\subsubsection{The model}
\begin{definition}[autoregressive conditional heteroscedastic model, general case]\index{autoregressive conditional heteroscedastic model}
	Let $a_t$ denotes the error terms. The ARCH(q) model assumes
	$$a_t = \sigma_t \epsilon_t$$
	where $\epsilon_t\sim WN(0,1)$, 
	and $$\sigma_t^2 = \alpha_0 + \sum_{i=1}^q \alpha_i a_{t-i}^2, \alpha_i\geq 0,\forall i=0,...,q$$
\end{definition}

\begin{definition}[autoregressive conditional heteroscedastic model, order 1, ARCH(1)]\index{autoregressive conditional heteroscedastic model}
Let $a_t$ denotes the error terms. The ARCH(1) model assumes
$$a_t = \sigma_t \epsilon_t$$
where $\epsilon_t\sim WN(0,1)$, 
and $$\sigma_t^2 = \alpha_0 + \alpha_1 a_{t-1}^2, \alpha_0,\alpha_1 \geq 0.$$

We also require $\alpha_1<1$ for stationarity.
\end{definition}

\begin{remark}[characteristics]\hfill
\begin{itemize}
	\item $a_t$ can be simply viewed as a 'special' white noise process in which the variance $E[a_t^2]$ is correlated with previous history.
	\item Note that $a_t$ is uncorrelated with previous history $a_{t-1}$ even though $\sigma_t$ is correlated, i.e. 
	$$E[a_ta_{t-1}] = E[\sigma_t\sigma_{t-1}\epsilon_t\epsilon_{t-1}] = E[\sigma_t\sigma_{t-1}]E[\epsilon_t\epsilon_{t-1}] = 0.$$
	\item $a_t$ is stationary if $\sigma_t$ is stationary. It can be showed that when $\alpha_0 > 0, 0 < \alpha_1 < 1$, $\sigma_t$ is stationary. Since $\epsilon_t$ and $\sigma_t$ are both stationary and independent, then $\sigma_t\epsilon_t$ is stationary.
\end{itemize}
\end{remark}

\begin{lemma}[weak stationarity of ARCH(1)]\label{ch:time-series-analysis:th:weakStationaryARCH(1)}
The ARCH(1) model given by
$$a_t = \sigma_t \epsilon_t, \sigma_t^2 = \alpha_0 + \alpha_1a_{t-1}^2, \alpha_1 \in [0,1)$$
has the following properties:
\begin{itemize}
	\item It has the constant mean $E[a_t] = 0$, and $$Var[a_t] = \frac{\alpha_0}{1 - \alpha_1},Cov(a_t,a_s) = \delta(s,t) \frac{\alpha_0}{1 - \alpha_1}.$$
	
	That is, ARCH(1) with $\alpha_1\in [0,1)$ is a weakly stationary process. 
	\item (stationary fourth moment) If $\alpha_1 < \sqrt{1/3}$, then $a_t$ has stationary fourth moment given by
	$$E[a_t^4] = 3\frac{\alpha_0^2 (1 + \alpha_1)}{(1-\alpha_1)(1-3\alpha_1^2)}.
	$$
	
\end{itemize}
\end{lemma}
\begin{proof}
(1)(a) $$E[a_t] = E[\sigma_t\epsilon_t]=E[\sigma_t]E[\epsilon_t] = 0.$$	
(b) 
Note that 
$$Cov(a_t,a_s) = E[a_t^2a_{s}^2] = E[\sigma_t\sigma_{s}\epsilon_t\epsilon_{s}] =E[\sigma_t\sigma_{s}]E[\epsilon_t^2]E[\epsilon_{s}^2] = 0, ~if~ t\neq s.$$
If $t=s$, $Cov(a_t,a_s) = E[\sigma_t^2]$
In the following we will calculate $E[\sigma_t^2]$.
Note that we can also write ARCH(1) as $$\sigma_t^2 = \alpha_0 + \alpha_1\epsilon_{t-1}^2 \sigma_{t-1}^2.$$
Then
\begin{align*}
E[\sigma_t^2] &= \alpha_0 + \alpha_1 E[\sigma_{t-1}^2] \\
&= \alpha_0 + \alpha_1(\alpha_0 + \alpha_1 E[\sigma_{t-2}^2]) \\
&= \alpha_0 + \alpha_1\alpha_0 + \alpha_1(\alpha_0 +  \alpha_1E[\sigma_{t-3}^2]) \\
&= \alpha_0(1 + \alpha_1 + \alpha_1^2 + ...)\\
&= \frac{\alpha_0}{1 - \alpha_1 }
\end{align*}
(2)
\begin{align*}
E[a_t^4] &= E[E[a_t^4|\cF_{t-1}]] \\
&=E[E[\epsilon_t^4(\alpha_0 + \alpha_1 a_{t-1}^2)^2|\cF_{t-1}]] \\
&=E[\epsilon_t^4E[(\alpha_0 + \alpha_1 a_{t-1}^2)^2|\cF_{t-1}]] \\
&=3E[E[(\alpha_0 + \alpha_1 a_{t-1}^2)^2|\cF_{t-1}]] \\
&=3E[(\alpha_0^2 + 2\alpha_0\alpha_1 a_{t-1}^2 + \alpha_1^2a_{t-1}^4)] \\
&=3(\alpha_0^2 + 2\alpha_0\alpha_1 E[a_{t-1}^2] + \alpha_1^2E[a_{t-1}^4] \\
&=3(\alpha_0^2 + 2\alpha_0\alpha_1 \frac{\alpha_0}{1 - \alpha_1} + \alpha_1^2E[a_{t-1}^4] \\
&= 3K + \beta E[a_{t-1}^4],K = \alpha_0^2 + 2\alpha_0\alpha_1 \frac{\alpha_0}{1 - \alpha_1} = \frac{\alpha_0(1 + \alpha_1)}{1 - \alpha_1},\beta = 3\alpha_1^2 \\
&= 3K + 3K\beta + \beta^2E[a_{t-2}^4] \\
&= 3K(1 + \beta + \beta^2 + ...) \\
&= \frac{3K}{1-\beta} 
\end{align*}
where we use $E[\epsilon_t^4] = 3$ .
\end{proof}






\begin{theorem}[basic statistical properties of ARCH(1)]\cite[118]{tsay2005analysis}\label{ch:time-series-analysis:th:BasicStatisticalProperitesARCH(1)}
	Consider an ARCH(1) model given by
	$$a_t = \sigma_t \epsilon_t, \sigma_t^2 = \alpha_0 + \alpha_1a_{t-1}^2, \alpha_1 \in [0,1).$$
Let $\cF_{t-1}$ denote the information up to time $t-1$. It follows that
\begin{itemize}
	\item (unconditional mean and variance)
	$$E[a_t] = 0, Var[a_t] = E[a_t^2] = \frac{\alpha_0}{1-\alpha_1},Cov(a_t,a_s) = \delta(s,t) \frac{\alpha_0}{1 - \alpha_1}.$$
	As a result, we can write
	$$\sigma_t^2 = (1-\alpha_1)Var[a_t] + \alpha_1 a_{t-1}^2.$$
	\item (conditional mean and variance) $a_t$ has conditional mean and variance given by
	$$E[a_t|\cF_{t-1}] = 0$$
	$$ E[a_t^2|\cF_{t-1}]  = Var[a_t|\cF_{t-1}] = \sigma_t^2 = (\alpha_0 + \alpha_1 a_{t-1}^2).$$
	$$ E[a_t^2|\cF_{t-2}]  = Var[a_t|\cF_{t-1}] = \sigma_t^2 = (\alpha_0 + \alpha_1 a_{t-1}^2).$$
	
	\item (covariance structure of squares)
	$$E(a_t^2,a_s^2) = $$
	$$Cov(a_t^2,a_s^2) = $$
	
	In particular$\alpha_0 = 0$, 
	\item (heavy tail property)
	$$kurt(a_t) = \frac{E[a_t^4]}{(E[a_t^2])^2} = 3\frac{1-\alpha_1^2}{1 - 3\alpha_1^2} > 3.$$
\end{itemize}
\end{theorem}
\begin{proof}
(1)(a) $$E[a_t] = E[\sigma_t\epsilon_t]=E[\sigma_t]E[\epsilon_t] = 0.$$
(b) \begin{align*}
Var[a_t] &= E[a_t^2] - (E[a_t])^2\\
&=E[a_t^2] \\
&=E[\sigma_t^2\epsilon_t^2]\\
&=E[\sigma_t^2]\\
&=\alpha_0 + \alpha_1E[a_{t-1}^2]
\end{align*}
use the fact that $a_t$ is a stationary process, thus $E[a_{t-1}^2] = E[a_t^2]$, we have $Var[a_t] = \frac{\alpha_0}{1-\alpha_1}$.\\
(2)
(a)
\begin{align*}
E[a_t|\cF_{t-1}] &= E[\epsilon_{t}\sqrt{\alpha_0 + \alpha_1 a_{t-1}^2}|\cF_{t-1}] \\
&= E[\epsilon_t|\cF_{t-1}]E[\sqrt{\alpha_0 + \alpha_1 a_{t-1}^2}|\cF_{t-1}] \\
&= 0
\end{align*}
(b)
\begin{align*}
Var[a_t|\cF_{t-1}] &= E[a_t^2|\cF_{t-1}] - (E[a_t|\cF_{t-1}])^2 \\
&= E[\epsilon_t^2|\cF_{t-1}](\alpha_0 + \alpha_1 a_{t-1}^2) - 0 \\
&= (\alpha_0 + \alpha_1 a_{t-1}^2)
\end{align*}
(3)
\begin{align*}
E[a_t^2a_s^2] &= E[\epsilon_t^2\epsilon_s^2\sigma_t^2\sigma_s^2] \\
&= E[\epsilon_t^2]E[\epsilon_s^2]E[\sigma_t^2\sigma_s^2] \\
&=E[(\alpha_0 + \alpha_1 a_{t-1}^2)(\alpha_0 + \alpha_1 a_{s-1}^2)]\\
&=E[(\alpha_0^2 + \alpha_1^2a_{t-1}^2a_{s-1}^2 + \alpha_0\alpha_1a_{t-1}^2 + \alpha_0\alpha_1a_{s-1}^2)]\\
&=\alpha_0^2 + 2\alpha_0\alpha_1\frac{\alpha_0}{1-\alpha_1} + \alpha_1^2E[a_{t-1}^2a_{s-1}^2] \\
&= K + \alpha_1^2  E[a_{t-1}^2a_{s-1}^2],K = \alpha_0^2 + 2\alpha_0\alpha_1\frac{\alpha_0}{1-\alpha_1} = \frac{\alpha_0^2(1+\alpha_1)}{1-\alpha_1}\\
& = \frac{K}{1-\alpha_1^2} \\
& = \frac{\alpha_0^2}{(1-\alpha_1)^2} 
\end{align*}


(4)Note that $E[a_t^2] = Var[a_t^2] = \alpha_0/(1-\alpha_1)$ and 
 $$E[a_t^4] = 3\frac{\alpha_0^2 (1 + \alpha_1)}{(1-\alpha_1)(1-3\alpha_1^2)},$$
 from \autoref{ch:time-series-analysis:th:weakStationaryARCH(1)}.
\end{proof}

\begin{remark}[interpretation]\cite[116]{tsay2005analysis}\hfill
\begin{itemize}
	\item (predictability) $a_t$ cannot be predicted based on history because of zero covariance; $a_t^2$ can be predicted from history because 
	$$E[a_t^2|\cF_{t-1}] = Var[a_t|\cF_{t-1}] = \sigma_t^2 = (\alpha_0 + \alpha_1 a_{t-1}^2). $$
	\item From $$\sigma_t^2 = (1-\alpha_1)Var[a_t] + \alpha_1 a_{t-1}^2,$$
	large variations in $a_t$ tend to be followed by large variations and small variations tend to be followed by small variations.
	\item $$kurt(a_t) = \frac{E[a_t^4]}{(E[a_t^2])^2} = \frac{E[\sigma_t^4]E[\epsilon_t^4]}{(E[\sigma_t^2]E[\epsilon_t^2])^2} = \frac{3E[\sigma_t^4]}{(E[\sigma_t^2])^2} > 3$$
	where we use the fact that $E[\sigma_t^4] > (E[\sigma_t^2])^2$ due to
	$$Var[a_t^2] = E[a_t^4] - (E[a_t^2])^2 > 0.$$
	\item  Intuitively, we can view $a_t$ as a mixture of Gaussian, which has fat tails.
\end{itemize}	
\end{remark}


\begin{theorem}[basic statistical properties of ARCH(q)]\cite[118]{tsay2005analysis}
	Consider a weakly stationary ARCH(q) model given by
	$$a_t = \sigma_t \epsilon_t, \sigma_t^2 = \alpha_0 + \alpha_1a_{t-1}^2 + \alpha_2 a_{t-2}^2 + \cdots + \alpha_q a_{t-q}^2.$$
	Let $\cF_{t-1}$ denote the information up to time $t-1$. It follows that
	\begin{itemize}
		\item (unconditional mean and variance)
		$$E[a_t] = 0, Var[a_t] = E[a_t^2] = \frac{\alpha_0}{1 - \sum_{i=1}^q\alpha_i},Cov(a_t,a_s) = \delta(s,t) \frac{\alpha_0}{1 - \sum_{i=1}^q\alpha_i}.$$
		\item (conditional mean and variance) $a_t$ has conditional mean and variance given by
		$$E[a_t|\cF_{t-1}] = 0, E[a_t^2|\cF_{t-1}]  = Var[a_t|\cF_{t-1}] = \sigma_t^2 = (\alpha_0 + \sum_{i=1}^q\alpha_i a_{t-i}^2).$$
		\item (mean prediction) Let $t < s$, then 
		$$E[a_t|\cF_s] = 0,$$
		that is, $a_t$ is unpredictable.
	\end{itemize}
\end{theorem}
\begin{proof}
	(1)(a) $$E[a_t] = E[\sigma_t\epsilon_t]=E[\sigma_t]E[\epsilon_t] = 0.$$
	(b) \begin{align*}
	Var[a_t] &= E[a_t^2] - (E[a_t])^2\\
	&=E[a_t^2] \\
	&=E[\sigma_t^2\epsilon_t^2]\\
	&=E[\sigma_t^2]\\
	&=\alpha_0 + \sum_{i=1}^q\alpha_i E[a_{t-i}^2]
	\end{align*}
	use the fact that $a_t$ is a stationary process, thus $E[a_{t-i}^2] = E[a_t^2]$, we have $Var[a_t] = \frac{\alpha_0}{1-\sum_{i=1}^q\alpha_i}$.\\
	(2)
	(a)
	\begin{align*}
	E[a_t|\cF_{t-1}] &= E[\epsilon_{t}\sqrt{\alpha_0 + \sum_{i=1}^q\alpha_i a_{t-i}^2}|\cF_{t-1}] \\
	&= E[\epsilon_t|\cF_{t-1}]E[\sqrt{\alpha_0 \sum_{i=1}^q\alpha_i a_{t-i}^2}|\cF_{t-1}] \\
	&= 0
	\end{align*}
	(b)
	\begin{align*}
	Var[a_t|\cF_{t-1}] &= E[a_t^2|\cF_{t-1}] - (E[a_t|\cF_{t-1}])^2 \\
	&= E[\epsilon_t^2|\cF_{t-1}](\alpha_0 + \sum_{i=1}^q\alpha_i a_{t-i}^2) - 0 \\
	&= (\alpha_0 + \sum_{i=1}^q\alpha_i a_{t-i}^2)
	\end{align*}
	(3) 
	$$E[a_t|\cF_s] = E[\sigma_t \epsilon_t|\cF_s] = E[\sigma_t|\cF_s]E[ \epsilon_t|\cF_s] = 0,$$
	note that $\sigma_t$ and $\epsilon_t$ are independent because $\sigma_t$ depends on the previous shocks that are independent of $\epsilon_t$.
\end{proof}


\subsubsection{Variance forecasting}

\begin{lemma}[conditional expectation equivalence between variance and square innovations for ARCH(q)]\label{ch:time-series-analysis:th:ConditionalExpectationEquivalenceVarianceSquareInnovationARCH(q)}
	Consider an ARCH(q) model given by
$$a_t = \sigma_t \epsilon_t, \sigma_t^2 = \alpha_0 + \sum_{i=1}^{q} \alpha_i a_{t-i}^2, \sum_{i=1}^{q}\alpha_i \in [0,1).$$	
		Let $\cF_t$ denote the information available up to time $t$. 
	Then
		$$E[\sigma_{t}^2|\cF_s] = E[a_{t}^2|\cF_s],s < t.$$
\end{lemma}
\begin{proof}
Note that $$a_t^2 = \sigma_t^2 \epsilon_t^2.$$
Take conditional expectation on both sides and get
	$$E[a_t^2|\cF_s] = E[\sigma_t^2 \epsilon_t^2|\cF_s] = E[\sigma_t^2|\cF_s] E[\epsilon_t^2|\cF_s] = E[\sigma_t^2|\cF_s].$$
where we used the fact that $E[\epsilon_t^2|\cF_s] = 1$ and the independence between $\sigma_t$ and $\epsilon_t$.	
\end{proof}


\begin{theorem}[variance prediction in ARCH(1)]
	Consider an ARCH(1) model given by
	$$a_t = \sigma_t \epsilon_t, \sigma_t^2 = \alpha_0 + \alpha_1a_{t-1}^2,\alpha_1 \in [0,1).$$
	
	Let $\cF_t$ denote the information available up to time $t$. Then,
	\begin{align*}
	E[a_{t+1}^2|\cF_t] = E[\sigma_{t+1}^2|\cF_t] \triangleq 
	\sigma^2_{t+1} &= \alpha_0 + \alpha_1 a_{t}^2 \\ 
	E[a_{t+2}^2|\cF_t] = E[\sigma_{t+2}^2|\cF_t]\triangleq \hat{\sigma}_{t+2}^2 
	&=\sigma^2 + (\alpha_1)(\sigma_{t+1}^2 - \sigma^2)\\
	E[a_{t+3}^2|\cF_t] = E[\sigma_{t+3}^2|\cF_t]\triangleq \hat{\sigma}_{t+3}^2 &= \sigma^2 + (\alpha_1)^2(\sigma_{t+1}^2 - \sigma^2)\\
	\vdots & \\
	E[a_{t+l}^2|\cF_t] = E[\sigma_{t+l}^2|\cF_t]\triangleq\hat{\sigma}_{t+l}^2 &= \sigma^2 + (\alpha_1)^{l-1}(\sigma_t^2 - \sigma^2)
	\end{align*}
	where $\sigma^2$ is the unconditional variance $$\sigma^2 = \frac{\alpha_0}{1 - \alpha_1}.$$	
\end{theorem}
\begin{proof}
Note that the equivalence between square innovation and variance is discussed in \autoref{ch:time-series-analysis:th:ConditionalExpectationEquivalenceVarianceSquareInnovationARCH(q)}. 	
Further note that $\sigma_{t+1}^2$ given $\cF_t$ is actually deterministic quantity. For the rest, we have
\begin{align*}
	\sigma_{t+1}^2 &= \alpha_0 + \alpha_1 a_{t}^2 \\
	E[\sigma_{t+2}^2|\cF_t]\triangleq \hat{\sigma}_{t+2}^2 &= \alpha_0 + \alpha_1 E[a_{t+1}^2|\cF_{t}] \\
	&=\alpha_0 + \alpha_1\sigma_{t+1}^2 \\
	&=\sigma^2 + (\alpha_1)(\sigma_{t+1}^2 - \sigma^2)\\
	E[\sigma_{t+3}^2|\cF_t]\triangleq \hat{\sigma}_{t+3}^2 &= \alpha_0 + \alpha_1 E[a_{t+2}^2|\cF_{t}] \\
	&=\alpha_0 + \alpha_1 \hat{\sigma}_{t+2}^2\\
	&=\sigma^2 + \alpha_1(\hat{\sigma}^2_{t+2}-\sigma^2)\\
	&=\sigma^2 + (\alpha_1)^2(\sigma_{t+1}^2 - \sigma^2)
	\end{align*}
where we use the fact of conditional variance(\autoref{ch:time-series-analysis:th:BasicStatisticalProperitesARCH(1)}) that
$$E[a_{t+1}^2|\cF_{t}] = \sigma_{t+1}^2 = (\alpha_0 + \alpha_1 a_{t}^2).$$
and the fact that $$E[a_{t+2}^2|\cF_t] = E[\sigma_{t+2}^2|\cF_t]\triangleq \hat{\sigma}_{t+2}^2$$.

	
Others can be proved similarly.	
\end{proof}

\begin{remark}[implication for convergence rate]
	We can see that $\hat{\sigma}_{t+l}^2 \to \sigma$ as $l\to \infty$. If the variance spikes up during a crisis, then the number of the period before the variance restore to equilibrium value $\sigma$ can be estimated using the value of $(\alpha_1+\beta_1)$(the larger the longer).
\end{remark}

\begin{theorem}[variance prediction in ARCH(q)]
	Consider an ARCH(q) model given by
	$$a_t = \sigma_t \epsilon_t, \sigma_t^2 = \alpha_0 + \sum_{i=1}^{q} \alpha_i a_{t-i}^2, \sum_{i=1}^{q}\alpha_i \in [0,1).$$
	
	Let $\cF_t$ denote the information available up to time $t$. Then,
	\begin{align*}
	E[a_{t+1}^2|\cF_t] = E[\sigma_{t+1}^2|\cF_t] \triangleq 
\sigma^2_{t+1} &= \alpha_0 + \sum_{i=1}^{q}\alpha_i a_{t-i+1}^2 \\ 
E[a_{t+2}^2|\cF_t] = E[\sigma_{t+2}^2|\cF_t]\triangleq \hat{\sigma}_{t+2}^2 
&=\alpha_0 + \alpha_1 \hat{\sigma}_{t+1}^2 + \sum_{i=2}^{q}\alpha_i a_{t-i+2}^2\\
E[a_{t+3}^2|\cF_t] = E[\sigma_{t+3}^2|\cF_t]\triangleq \hat{\sigma}_{t+3}^2 &= \alpha_0 + \sum_{i=1}^{2}\alpha_i \hat{\sigma}_{t-i+3}^2 + \sum_{i=3}^{q}\alpha_i a_{t-i+3}^2\\
\vdots & \\
E[a_{t+l}^2|\cF_t] = E[\sigma_{t+l}^2|\cF_t]\triangleq\hat{\sigma}_{t+l}^2 &= \alpha_0 + \sum_{i=1}^{l-1}\alpha_i \hat{\sigma}_{t-i+l}^2 + \sum_{i=l}^{q}\alpha_i a_{t-i+l}^2\\
\vdots & \\
E[a_{t+\infty}^2|\cF_t] = E[\sigma_{t+\infty}^2|\cF_t]\triangleq \hat{\sigma}_{t+\infty}^2 &= \frac{\alpha_0}{1 - \sum_{i=1}^{q}\alpha_i} 
	\end{align*}
	where note that the unconditional variance $$\sigma^2 = \frac{\alpha_0}{1 - \sum_{i=1}^{q}\alpha_i}.$$	
\end{theorem}
\begin{proof}
	\begin{align*}
	\sigma_t^2 &= \alpha_0 + \sum_{i=1}^{q}\alpha_i a_{t-i}^2 \\
	E[\sigma_{t+1}^2|\cF_t]\triangleq \hat{\sigma}_{t+1}^2 &= \alpha_0 + \sum_{i=1}^{q}\alpha_i E[a_{t-i+1}^2|\cF_{t}] \\
	&=\alpha_0 + \sum_{i=1}^{q}\alpha_i a_{t-i+1}^2 \\
	E[\sigma_{t+2}^2|\cF_t]\triangleq \hat{\sigma}_{t+2}^2 &= \alpha_0 + \alpha_1 E[a_{t-i+2}^2|\cF_{t}] +\sum_{i=2}^{q}\alpha_i E[a_{t-i+2}^2|\cF_{t}] \\
	&= \alpha_0 + \alpha_1 \hat{a}_{t-i+1}^2 +\sum_{i=2}^{q}\alpha_i a_{t-i+1}^2 \\
	E[\sigma_{t+3}^2|\cF_t]\triangleq \hat{\sigma}_{t+3}^2 &= \alpha_0 + \sum_{i=1}^{2}\alpha_i E[a_{t-i+3}^2|\cF_{t}] +\sum_{i=3}^{q}\alpha_i E[a_{t-i+3}^2|\cF_{t}] \\
	&= \alpha_0 + \sum_{i=1}^{2}\alpha_i \hat{\sigma}_{t-i+3}^2 +\sum_{i=3}^{q}\alpha_i a_{t-i+3}^2 
	\end{align*}
	Others can be proved similarly.	
	
	To calculate the variance prediction for $l\to\infty$, we use the result of unconditional variance.
\end{proof}

\subsubsection{Detect ARCH effect}


\begin{lemma}[order determination in ARCH process via linear regression]
	
	$$\sigma_t^2 = \alpha_0 + \alpha_1 a_{t-1}^2 + \cdots + \alpha_m a_{t-m}^2$$
	
	
\end{lemma}
\begin{proof}
Note that in our $ARCH(m)$ model, we have	
	$$\sigma_t^2 = \alpha_0 + \alpha_1 a_{t-1}^2 + \cdots + \alpha_m a_{t-m}^2.$$
Because $a_t^2$ is the unbiased estimator of $\sigma_t^2$, i.e. 
$$E[a_t^2] = E[\sigma_t^2\epsilon_t^2] = E[\sigma_t^2]$$	

\end{proof}


\subsubsection{Parameter estimation}
\begin{lemma}[conditional likelihood estimation ]\cite[120]{tsay2005analysis}
	
The conditional log-likelihood function is given by
$$L = \sum_{t=m+1}^T[-\frac{1}{2}\ln (2\pi) - \frac{1}{2}\ln (\sigma_t^2) - \frac{1}{2}\frac{a_t^2}{\sigma_t^2}]$$	
\end{lemma}

\subsection{GARCH models}
\subsubsection{The model}
\begin{definition}[generalized autoregressive conditional heteroscedastic model, general case]\index{autoregressive conditional heteroscedastic model}\cite[132]{tsay2005analysis}
	Let $a_t$ denotes the error terms. The GARCH(p,q) model assumes
	$$a_t = \sigma_t \epsilon_t$$
	where $\epsilon_t\sim WN(0,1)$, 
	and $$\sigma_t^2 = \alpha_0 + \sum_{i=1}^p \alpha_i a_{t-i}^2 + \sum_{i=1}^q \beta_i \sigma_{t-i}^2, \alpha_i, \beta_j \geq 0,\forall i,j \geq 1$$
\end{definition}


\begin{definition}[generalized autoregressive conditional heteroscedastic model, order 1, GARCH(1,1)]\index{autoregressive conditional heteroscedastic model}
	Let $a_t$ denotes the error terms. The GARCH(q) model assumes
	$$a_t = \sigma_t \epsilon_t$$
	where $\epsilon_t\sim WN(0,1)$, 
	and $$\sigma_t^2 = \alpha_0 + \alpha_1 a_{t-1}^2+ \beta_1 \sigma_{t-1}^2, \alpha_0,\alpha_1,\beta_1 \geq 0.$$
	
	We also require $\alpha_1 + \beta_1<1$ for stationarity.
\end{definition}


\begin{lemma}[weak stationarity of GARCH(1,1)]	\label{ch:time-series-analysis:th:weakStationaryGARCH(1)}
	Consider a GARCH(1,1) model given by
	$$a_t = \sigma_t \epsilon_t, \sigma_t^2 = \alpha_0 + \alpha_1a_{t-1}^2 + \beta_1 \sigma_{t-1}^2, \alpha_1 + \beta_1 \in [0,1).$$
	
	It follows that
	\begin{itemize}
		\item It has the constant mean $E[a_t] = 0$, and $$Cov(a_t,a_s) = \delta(s,t) \frac{\alpha_0}{1 - \alpha_1 - \beta_1}.$$
		That is, GARCH(1,1) with $\alpha_1+\beta_1\in [0,1)$ is a weakly stationary process. 
		\item (stationary fourth moment) 
	\end{itemize}
\end{lemma}
\begin{proof}
	(1) $$E[a_t] = E[\sigma_t\epsilon_t]=E[\sigma_t]E[\epsilon_t] = 0.$$	
	(2) 
	Note that 
	$$Cov(a_t,a_s) = E[a_ta_{s}] = E[\sigma_t\sigma_{s}\epsilon_t\epsilon_{s}] =E[\sigma_t\sigma_{s}]E[\epsilon_t]E[\epsilon_{s}] = 0, ~if~ t\neq s.$$
	where we use the independence between $\epsilon_t$ and $\sigma_t$.
	If $t=s$, $Cov(a_t,a_s) = E[\sigma_t^2]E[\epsilon_t^2] = E[\sigma_t^2]$.
	In the following we will calculate $E[\sigma_t^2]$.
	Note that we can also write GARCH(1,1) as $$\sigma_t^2 = \alpha_0 + \alpha_1\epsilon_{t-1}^2 \sigma_{t-1}^2+\beta_1\epsilon_{t-1}^2 \sigma_{t-1}^2.$$
	Then
	\begin{align*}
	E[\sigma_t^2] &= \alpha_0 + (\alpha_1 + \beta_1) E[\sigma_{t-1}^2] \\
	&= \alpha_0 + (\alpha_1+\beta_1)(\alpha_0 + (\alpha_1+\beta_1) E[\sigma_{t-2}^2]) \\
	&= \alpha_0 + (\alpha_1+\beta_1)\alpha_0 + (\alpha_1+\beta_1)(\alpha_0 +  (\alpha_1+\beta_1)E[\sigma_{t-3}^2]) \\
	&= \alpha_0(1 + (\alpha_1+\beta_1) + (\alpha_1+\beta_1)^2 + ...)\\
	&= \frac{\alpha_0}{1 - \alpha_1 - \beta_1}
	\end{align*}
	
	
\end{proof}

\begin{theorem}[basic statistical properties of GARCH(1,1)]\cite[132]{tsay2005analysis}\label{ch:time-series-analysis:th:basicStatisticalPropertiesGARCH(1,1)}
	The GARCH(1,1) model given by
	$$a_t = \sigma_t \epsilon_t, \sigma_t^2 = \alpha_0 + \alpha_1a_{t-1}^2 + \beta_1\sigma_{t-1}^2,\alpha_1+\beta_1\in [0,1)$$
	has the following property:
\begin{itemize}
	\item (unconditional mean and variance) 
		$$E[a_t] = 0$$
	and
	$$Var[a_t] = \frac{\alpha_0}{1-\alpha_1-\beta_1}.$$
	\item (conditional mean and variance) $a_t$ has conditional mean and variance given by
	$$E[a_t|\cF_{t-1}] = 0,$$ 
	$$E[a_t^2|\cF_{t-1}] = Var[a_t|\cF_{t-1}] = \sigma_t^2 = (\alpha_0 + \alpha_1 a_{t-1}^2 + \beta_1 \sigma_{t-1}^2).$$
		\item (mean prediction) Let $t < s$, then 
$$E[a_t|\cF_s] = 0,$$
that is, $a_t$ is unpredictable.
\end{itemize}	
\end{theorem}
\begin{proof}
(1) $$E[a_t] = E[\sigma_t\epsilon_t]=E[\sigma_t]E[\epsilon_t] = 0.$$
 \begin{align*}
Var[a_t] &= E[a_t^2] - (E[a_t])^2\\
&=E[a_t^2] \\
&=E[\sigma_t^2\epsilon_t^2]\\
&=E[\sigma_t^2]\\
&=\alpha_0 + \alpha_1E[a_{t-1}^2] + \beta_1\sigma_{t-1}^2\\
&=\alpha_0 + (\alpha_1 + \beta_1)E[a_{t-1}^2]
\end{align*}
use the fact that $a_t$ is a stationary process, thus $E[a_{t-1}^2] = E[a_t^2]$, we have $Var[a_t] = \frac{\alpha_0}{1-\alpha_1-\beta_1}$.	
(2)
(a)
\begin{align*}
E[a_t|\cF_{t-1}] &= E[\epsilon_{t}\sqrt{\alpha_0 + \alpha_1 a_{t-1}^2 + \beta_1 \sigma_{t-1}^2}|\cF_{t-1}] \\
&= E[\epsilon_t|\cF_{t-1}]E[\sqrt{\alpha_0 + \alpha_1 a_{t-1}^2 + \beta_1 \sigma_{t-1}^2}|\cF_{t-1}] \\
&= 0
\end{align*}
(b)
\begin{align*}
Var[a_t|\cF_{t-1}] &= E[a_t^2|\cF_{t-1}] - (E[a_t|\cF_{t-1}])^2 \\
&= E[\epsilon_t^2|\cF_{t-1}](\alpha_0 + \alpha_1 a_{t-1}^2 + \beta_1 \sigma_{t-1}^2) - 0 \\
&= (\alpha_0 + \alpha_1 a_{t-1}^2 + \beta_1 \sigma_{t-1}^2)
\end{align*}
	(3) 
$$E[a_t|\cF_s] = E[\sigma_t \epsilon_t|\cF_s] = E[\sigma_t|\cF_s]E[ \epsilon_t|\cF_s] = 0,$$
note that $\sigma_t$ and $\epsilon_t$ are independent because $\sigma_t$ depends on the previous shocks that are independent of $\epsilon_t$.
\end{proof}


\begin{note}[$ARCH(\infty)$ representation of $GARCH(1,1)$]
\begin{align*}
\sigma_t^2 &= \alpha_0 + \alpha_1 a_{t-1}^2 + \beta_1\sigma_{t-1}^2\\
&=\alpha_0 + \alpha_1 a_{t-1}^2 + \beta_1(\alpha_0 + \alpha_1 a_{t-2}^2 + \beta_1\sigma_{t-2}^2) \\
&=\alpha_0 + \alpha_1 a_{t-1}^2 + \beta_1\alpha_0 + \beta_1\alpha_1 a_{t-2}^2 + \beta_1^2\sigma_{t-2}^2\\
&=\alpha_0 + \alpha_1 a_{t-1}^2 + \beta_1\alpha_0 + \beta_1\alpha_1 a_{t-2}^2 + \beta_1^2(\alpha_0 + \alpha_1 a_{t-2}^2 + \beta_1\sigma_{t-2}^2) \\
&\vdots \\
&=\frac{\alpha_0}{1 - \beta_1} + \alpha_1\sum_{i=0}^{\infty} a_{t-1-i}^2\beta_1^i
\end{align*}
Therefore, $\sigma_t^2$ contains history back to infinity, even though the strength decreases geometrically. 
\end{note}


\subsubsection{Variance forecasting}
\begin{lemma}[conditional expectation equivalence between variance and square innovations for ARCH(q)]\label{ch:time-series-analysis:th:ConditionalExpectationEquivalenceVarianceSquareInnovationGARCH(p,q)}
	Consider an GARCH(p,q) model given by
		$$a_t = \sigma_t \epsilon_t$$
	where $\epsilon_t\sim WN(0,1)$, 
	and $$\sigma_t^2 = \alpha_0 + \sum_{i=1}^p \alpha_i a_{t-i}^2 + \sum_{i=1}^q \beta_i \sigma_{t-i}^2, \alpha_i, \beta_j \geq 0,\forall i,j \geq 1.$$	
	Let $\cF_t$ denote the information available up to time $t$. 
	Then
	$$E[\sigma_{t}^2|\cF_s] = E[a_{t}^2|\cF_s],s < t.$$
\end{lemma}
\begin{proof}
	Note that $$a_t^2 = \sigma_t^2 \epsilon_t^2.$$
	Take conditional expectation on both sides and get
	$$E[a_t^2|\cF_s] = E[\sigma_t^2 \epsilon_t^2|\cF_s] = E[\sigma_t^2|\cF_s] E[\epsilon_t^2|\cF_s] = E[\sigma_t^2|\cF_s].$$
	where we used the fact that $E[\epsilon_t^2|\cF_s] = 1$ and the independence between $\sigma_t$ and $\epsilon_t$.	
\end{proof}

\begin{theorem}[variance prediction in GARCH(1,1)]
Consider a GARCH(1,1) model given by
$$a_t = \sigma_t \epsilon_t, \sigma_t^2 = \alpha_0 + \alpha_1a_{t-1}^2 + \beta_1\sigma_{t-1}^2,\alpha_1+\beta_1\in [0,1).$$

	
	Let $\cF_t$ denote the information available up to time $t$. Then,
	\begin{align*}
	E[\sigma_{t+1}^2|\cF_t]\triangleq \sigma_{t+1}^2 
	&=\alpha_0 + \alpha_1 a_{t}^2  + \beta_1 \sigma_{t}^2\\
	E[\sigma_{t+2}^2|\cF_t]\triangleq \hat{\sigma}_{t+2}^2 
	&=\sigma^2 + (\alpha_1+ \beta_1)(\sigma_{t+1}^2 - \sigma^2)\\
	E[\sigma_{t+3}^2|\cF_t]\triangleq \hat{\sigma}_{t+2}^2 &= \sigma^2 + (\alpha_1+ \beta_1)^2(\sigma_{t+1}^2 - \sigma^2)\\
	\vdots & \\
	E[\sigma_{t+l}^2|\cF_t]\triangleq\hat{\sigma}_{t+l}^2 &= \sigma^2 + (\alpha_1+ \beta_1)^{l-1}(\sigma_{t+1}^2 - \sigma^2)
	\end{align*}
where $\sigma^2$ is the unconditional variance $$\sigma^2 = \frac{\alpha_0}{1 - \alpha_1 - \beta_1}.$$	
\end{theorem}
\begin{proof}
Note that the equivalence between square innovation and variance is discussed in \autoref{ch:time-series-analysis:th:ConditionalExpectationEquivalenceVarianceSquareInnovationARCH(q)}.

Further note that $\sigma_{t+1}^2$ given $\cF_t$ is actually deterministic quantity. For the rest, we have
\begin{align*}
\sigma_{t+1}^2 &= \alpha_0 + \alpha_1 a_{t}^2  + \beta_1 \sigma_{t}^2 \\
E[\sigma_{t+2}^2|\cF_t]\triangleq \hat{\sigma}_{t+2}^2 &= \alpha_0 + \alpha_1 E[a_{t+1}^2|\cF_{t}] + \beta_1 \sigma_{t+1}^2 \\
&=\alpha_0 + \alpha_1\sigma_{t+1}^2 + \beta_1\sigma_{t+1}^2\\
&=\alpha_0 + (\alpha_1+ \beta_1)\sigma_{t+1}^2\\
&=\sigma^2 + (\alpha_1+ \beta_1)(\sigma_{t+1}^2 - \sigma^2)\\
E[\sigma_{t+3}^2|\cF_t]\triangleq \hat{\sigma}_{t+3}^2 &= \alpha_0 + \alpha_1 E[a_{t+2}^2|\cF_{t}] + \beta_1 \hat{\sigma}_{t+2}^2 \\
&=\alpha_0 + \alpha_1 \hat{\sigma}_{t+2}^2 + \beta_1\hat{\sigma}_{t+2}^2\\
&=\alpha_0 + (\alpha_1+ \beta_1)\hat{\sigma}_{t+2}^2\\
&=\sigma^2 + (\alpha_1+ \beta_1)(\hat{\sigma}_{t+2}-\sigma^2)\\
&=\sigma^2 + (\alpha_1+ \beta_1)^2(\sigma_{t+1}^2 - \sigma^2)
\end{align*}	
where we use the fact(\autoref{ch:time-series-analysis:th:basicStatisticalPropertiesGARCH(1,1)}) that
	$$E[a_{t+1}^2|\cF_{t}] = \sigma_{t+1}^2 = (\alpha_0 + \alpha_1 a_{t}^2 + \beta_1 \sigma_{t}^2).$$

Others can be proved similarly.
\end{proof}

\begin{remark}[implication for convergence rate]
	We can see that $\hat{\sigma}_{t+l}^2 \to \sigma$ as $l\to \infty$. If the variance spikes up during a crisis, then the number of the period before the variance restore to equilibrium value $\sigma$ can be estimated using the value of $(\alpha_1+\beta_1)$(the larger the longer).
\end{remark}



\subsubsection{Detect GARCH effect}


\subsubsection{Parameter estimation}


\subsection{ARMA-GARCH model}

\begin{definition}[ARMA(1,1)-GARCH(1,1) model]
	
	$$X_t = \mu + \phi X_{t-1} + $$
\end{definition}



see here for parameter estimation \href{https://en.wikipedia.org/wiki/Autoregressive_conditional_heteroskedasticity}{link}

\section{Notes on Bibliography}


Introductory level treatment, see \cite{chatfield2003analysis}.
Intermediate level treatment, see \cite{brockwell2002introduction}\cite{hamilton1994time}\cite{shumway2010time}\cite{enders2014applied}\cite{subbaRao2017timeSeries}.
Advanced level treatment, see \cite{brockwell1991time}


For multivariate time series and cointegration, see \cite{tsay2013multivariate}\cite{pfaff2008analysis}\cite{lutkepohl2005new}.

For finanical time series, see \cite{tsay2005analysis}\cite{tsay2013multivariate}\cite{hayashi2000econometrics}.


Analysis of Integrated and Cointegrated Time Series with R




\printbibliography

\chapter{Time series analysis II}

\section{Multivariate time series}
\begin{definition}[multivariate time series]\index{multivariate time series}
A $K$ dimensional \textbf{multivariate time series }$\{X_t\}_{t\in T}$ is a collection of $K$ dimensional random vectors $X_t$ taking values in $\R^K$ whose the indexing set $T$ is the set of integers.
\end{definition}


\begin{definition}[mean vector, covariance matrix, cross-correlation matrix]\index{mean vector}\index{covariance matrix}\index{cross-correlation matrix}
Given a $K$ dimensional multivariate time series, 
the mean vector is defined as
$$\mu(t) = E[X_t],\mu(t)\in \R^K.$$
The covariance matrix is defined as
$$\Gamma_0(t) = E[(X_t - \mu)(X_t - \mu)^T], \Gamma_0(t) \in \R^{K\times K}.$$
And the cross-correlation matrix
$$\Gamma_k(t) = Cov(X_t,X_{t-k}) = E[(X_t - \mu)(X_{t-k} - \mu)^T],\Gamma_k(t) \in \R^{K\times K}.$$
\end{definition}


\begin{definition}[weakly stationary]\index{weakly stationary}
The multivariate time series $X_t$ is weakly stationary if
$\mu(t)$ is constant and $\Gamma_k(t)$ does not depend on $t$.
\end{definition}

\begin{definition}[Matrix polynomial of lag operator]
A matrix polynomial  of lag operator of order $p$is defined as
$$\Theta_p(B) = I_k + \Theta_1B + \Theta_2 B^2 + ... + \Theta_p B^p$$
where $B$ is the lag operator(we can view as a special scalar), $I_k,\Theta_i \in \R^{k\times k}$
\end{definition}


\begin{definition}[invertibility]\cite[7]{tsay2013multivariate}
A multivariate(m dimension) time series $z_t$ is said to be \textbf{invertible} if it can be written as
$$z_t = c + a_t + \sum_{j=1}^\infty \pi_j z_{t-j},$$
where
\begin{itemize}
	\item $c\in \R^m $ is a constant vector;
	\item $\pi_i \in \R^{m\times m}, i=1,2,...$ are matrices;
	\item $\{a_t\}$ is a white noise process with $Var[a_t] = \Sigma_a \in \R^{m\times m}$. 
\end{itemize}

To ensure convergence, we also require $\pi_i\to 0$ as $i\to \infty$.
\end{definition}


\section{Vector autoregressive models}
\subsection{Models}



\subsubsection{VAR(1) model}
\begin{definition}[VAR(1)]
An $m$ dimensional multivariate time series $X_t$ is called $VAR(1)$ process if 
$$X_t = C + \Phi_1 X_{t-1} + \eta_t$$
where $X_t\in \R^m,\Phi_i\in \R^{m\times m}, \eta_t\sim MN(0,\Sigma),\Sigma \in \R^{m\times m}, \eta_t, C \in \R^m$.

\end{definition}

\begin{example}
A bivariate VAR(1) model is given explicitly by
$$\begin{bmatrix}
X_{1,t} \\
X_{2,t}
\end{bmatrix} = \begin{bmatrix}
C_1\\
C_2
\end{bmatrix} + \begin{bmatrix}
\Phi_{1,11} & \Phi_{1,12}\\
\Phi_{1,21} & \Phi_{1,22}
\end{bmatrix}\begin{bmatrix}
X_{1,t-1}\\
X_{2,t-1}  
\end{bmatrix}$$	
\end{example}

\begin{lemma}[mean-adjusted of VAR(1)]
	A stable VAR(1) process $X_t = C + A X_{t-1} + u_t$  can be written as
	$$X_t - \mu = A(X_{t-1} - \mu) + u_t$$
	where $\mu = (I_K - A)^{-1}C$.
\end{lemma}
\begin{proof}
	$$X_t - \mu = A(X_{t-1} - \mu) + u_t \implies X_t - (I - A)\mu = AX_{t-1} + u_t.$$
	Therefore, we can set $(I-A)\mu = C$. 
\end{proof}







\begin{lemma}[stationary condition for VAR(1) process]\cite[32]{tsay2013multivariate}\label{ch:time-series-analysis:th:stationaryConditionForVAR(1)}
A VAR(1) process $$X_t = C + A X_{t-1} + u_t$$
is stationary if all eigenvalues of $A$ satisfy $
\abs{\lambda} < 1$. 	
Or equivalent 
$$det(I_K - Az) \neq 0, \forall z\in \C,  \abs{z} \leq 1. $$

Such VAR(1) process is said to be \textbf{stationary or stable}.
\end{lemma}
\begin{proof}
(1)
\begin{align*}
X_t &= AX_{t-1} + u_t \\
&= A(AX_{t-2} + u_{t-1}) + u_t \\
&= \vdots \\
&= A^{t-v}X_{v} + \sum_{i=0}^{t-1} A^i u_{t-i}
\end{align*}
As $v\to -\infty$, $A^\infty$ will have a limit if the spectral radius of $A$ is smaller than 1(\autoref{ch:linearalgebra:th:convergenceofmatrixpower}). 
(2)
To show that equivalence, suppose there exists a $z_0$ such that $\abs{z_0} \leq 1$ and $det(I_K - Az_0) = 0$. 
Then
$$det(I_K - Az_0) = z_0^K det(I_k/z_0 - A) = 0;$$
that is, there exists $1/z_0 = \lambda_0, \abs{\lambda_0} \geq 1$ as the eigenvalue of $A$.
\end{proof}

\begin{note}[implications of stability]
From \autoref{ch:linearalgebra:singularityfromspectralradius}, a stable VAR(1) process will ensure
$(I_K - A)^{-1}$
exist; and
$$(I_K - A)^{-1} = I_K + A + A^2 + ...$$
Then
$$X_t = (I_K-AB)^{-1}(C + u_t),$$
where $B$ is the lag operator.
\end{note}

\begin{lemma}[Vector moving averaging representation of stable VAR(1)]\label{ch:time-series-analysis:th:VectorMARepresentationStableVAR(1)}\cite[36]{tsay2013multivariate}
	A stable $K$ dimensional VAR(1) process 
	$$X_t = C + A X_{t-1} + u_t$$
	has the moving averaging representation as
	$$X_t = (I_K + A + A^2 + ... )C + (I_Ku_t + Au_{t-1} + A^2u_{t-2} + ...),$$
	or equivalently,
	$$X_t = (I_K -A)^{-1}C + (I_K-AB)^{-1}u_t,$$
where $B$ is the lag operator.	
\end{lemma}
\begin{proof}
	Directly recursively expand the original equation.
\end{proof}



\begin{lemma}[basic properties of a stable VAR(1)]\cite[16]{lutkepohl2005new}
A \textbf{stable} $K$ dimensional VAR(1) process 
$$X_t = C + A X_{t-1} + u_t$$
has
\begin{itemize}
\item constant mean $\mu = (I_K - A)^{-1}C$.
\item constant covariance variance: $Cov(X_t,X_t) = E[(X_t - \mu)^2] = (I_K - A)^{-1}Cov(\mu_t,\mu_t)(I_K - A)^{-T} = \sum_{i=0}^\infty A^{h+i} Cov(u_t,u_t) (A^i)^T.$
\item shift-invariant cross-covariance matrix: $$\Gamma(h) = \sum_{i=0}^\infty A^{h+i} \Sigma_u (A^i)^T.$$

\item it is a weakly stationary process; moreover, since $u_t$ is multivariate Gaussian, it is also a strongly stationary process. 
\end{itemize}
\end{lemma}
\begin{proof}
(1)(2) use moving average representation (\autoref{ch:time-series-analysis:th:VectorMARepresentationStableVAR(1)}) to prove. 
\begin{align*}
X_t &= (I_K + A + A^2 + ... )C + (I_Ku_t + Au_{t-1} + A^2u_{t-2} + ...) \\
X_t &= (I_K-A)^{-1}C + (I_K-A)^{-1}u_t \implies E[X_t] = (I_K-A)^{-1}C
\end{align*}

(3)
Note that
$$X_t - \mu = (I_K - AB)^{-1}u_t.$$

Then,
$$Cov(X_{t+h},X_t) = E[(X_{t+h} - \mu)(X_t - \mu)^T] = A^hE[(X_t - \mu)^(X_t - \mu)^T].$$
(4) directly from (1)(2)(3).
\end{proof}


\subsubsection{VAR(2) model}
\begin{definition}[VAR(2)]
	An $m$ dimensional multivariate time series $X_t$ is called $VAR(2)$ process if 
	$$X_t = C + \Phi_1 X_{t-1} + \Phi_2 X_{t-2} + \eta_t$$
	where $X_t\in \R^m,\Phi_1,\Phi_2 \in \R^{m\times m}, \eta_t\sim MN(0,\Sigma),\Sigma \in \R^{m\times m}, \eta_t, C \in \R^m$.
\end{definition}


\begin{lemma}[VAR(1) representation of VAR(2) process]
	Every $m$ dimensional $VAR(2)$ process $X_t$ is equivalent to a $2m$ dimensional $VAR(1)$ process by using the following transformation procedures:
	Define
	\begin{align*}
	Z_t &= (X_t^T,X_{t-1}^T)^T\\
	Z_{t-1} &= (X_{t-1}^T,X_{t-2}^T)^T
	\end{align*}
	where $Z_t \in \R^{2m}$. Then
	$$Z_t = D + AZ_{t-1} + F$$
	where $D\in \R^{2m},A\in \R^{2m\times 2m},F\in \R^{2m}$ are given as
	$$
	D = \begin{bmatrix}
	C\\ 0_m 
	\end{bmatrix}
	,
	A = \begin{pmatrix}
	\Phi_1 & \Phi_2 \\ 
	I_m & 0 
	\end{pmatrix}
	,
	F = \begin{bmatrix}
	\eta_t\\ 0_m 
	\end{bmatrix}
	$$ 
\end{lemma}


\begin{lemma}[stability condition for VAR(2) process]\cite[38]{tsay2013multivariate}
	The stability condition for a VAR(2) process is
	$$det(I_{2m} - \phi_1z - \phi_2 z^2) \neq 0, \forall \abs{z} \leq 1.$$
\end{lemma}
\begin{proof}
Use the equivalent representation VAR(1) of VAR(2) and stability condition for VAR(1) \autoref{ch:time-series-analysis:th:stationaryConditionForVAR(1)}.
We have
\begin{align*}
det(I_{2m} - Az) &= \begin{vmatrix}
I_m - \phi_1 z & -\phi_2 z\\
-I_m z  & I_m
\end{vmatrix} \\
& =  \begin{vmatrix}
I_m - \phi_1 z - \phi_2 z^2 & -\phi_2 z\\
0  & I_m
\end{vmatrix} \\
& = \abs{I_m - \phi_1 z - \phi_2 z^2}
\end{align*}
where we multiply the second column block matrix by $z$ and add to the first column(which will not change the determinant value(\autoref{appendix:th:matrixDeterminantProperties})).
\end{proof}




\subsubsection{VAR(p) model}
\begin{definition}[VAR operator]\index{VAR operator}
	A VAR operator of order $q$ is a matrix polynomial of lag operator given as
	$$\Phi_p(B) = I_k - \Phi_1B - \Phi_2 B^2 - ... - \Phi_p B^p$$
	where $B$ is the lag operator, $\Theta_i \in \R^{k\times k}$
\end{definition}





\begin{definition}[VAR(p) process]\cite[27]{tsay2005analysis}
	An $m$ dimensional multivariate time series $X_t$ is called $VAR(p)$ process if 
	$$X_t = C + \Phi_1 X_{t-1} + \Phi_2 X_{t-2} + ... + \Phi_pX_{t-p} + \eta_t$$
	where $X_t\in \R^m,\Phi_i\in \R^{m\times m}, \eta_t\sim MN(0,\Sigma),\Sigma \in \R^{m\times m}, \eta_t, C \in \R^m$.
	Using VAR operator, we have
	$$\Phi_p(B) X_t = C + \eta_t$$
\end{definition}


\begin{lemma}[VAR(1) representation of VAR(p) process]
Every $m$ dimensional $VAR(p)$ process $X_t$ is equilvalent to a $mp$ dimensional $VAR(1)$ process by using the following transformation procedures:
Define
\begin{align*}
Z_t &= (X_t^T,X_{t-1}^T,...,X_{t-p+1}^T)^T\\
Z_{t-1} &= (X_{t-1}^T,X_{t-2}^T,...,X_{t-p}^T)^T
\end{align*}
where $Z_t \in \R^{m\times p}$. Then
$$Z_t = D + AZ_{t-1} + F$$
where $D\in \R^{mp},A\in \R^{mp\times mp},F\in \R^{mp}$ are given as
$$
D = \begin{bmatrix}
	C\\ 0_m \\ 0_m \\ \vdots \\ 0_m \\0_m
\end{bmatrix}
,
A = \begin{pmatrix}
	\phi_1 & \phi_2 & \phi_3 & \dots & \dots & \phi_p\\ 
	I_m & 0 & 0 & \dots & \dots & 0 \\ 
	0 & I_m & 0 & \dots & \dots & 0 \\ 
	\vdots & \ddots & \ddots &  &  & 0\\ 
	0 & 0 & \ddots & I_m & 0 & 0\\ 
	0 & 0 &  \dots & 0 & I_m & 0
\end{pmatrix}
,
F = \begin{bmatrix}
	\eta_t\\ 0_m \\ 0_m \\ \vdots \\ 0_m \\0_m
\end{bmatrix}
$$ 
\end{lemma}


\begin{lemma}[stability condition for VAR(p) process]
The stability condition for a VAR(p) process is
$$det(I_m - \phi_1z - ... - \phi_p z^p) \neq 0, \forall \abs{z} \leq 1.$$
or equivalently, $$det(I_{mp} - \bm{A}z)\neq 0, \forall \abs{z} \leq 1.$$
\end{lemma}
\begin{proof}
Use the VAR(1) representation and stability condition for VAR(1) process. Note that
\begin{align*}
det(I_{mp} - Az) &= \begin{vmatrix}
I_m-\phi_1 z & -\phi_2 z & -\phi_3 z& \dots & \dots & -\phi_p z\\ 
-I_m z & I_m & 0 & \dots & \dots & 0 \\ 
0 & -I_m z& I_m & \dots & \dots & 0 \\ 
\vdots & \ddots & \ddots &  &  & 0\\ 
0 & 0 & \ddots & -I_m z& I_m & 0\\ 
0 & 0 &  \dots & 0 & -I_m z & I_m
\end{vmatrix} \\
&=\begin{vmatrix}
I_m-\phi_1 z -\phi_2 z^2 - \cdots \phi_p z^p & -\phi_2 z - \phi_3z^2 -\cdots & -\phi_3 z& \dots & \dots & -\phi_p z\\ 
0 & I_m & 0 & \dots & \dots & 0 \\ 
0 & & I_m & \dots & \dots & 0 \\ 
\vdots & \ddots & \ddots &  &  & 0\\ 
0 & 0 & \ddots & 0& I_m & 0\\ 
0 & 0 &  \dots & 0 & 0 & I_m
\end{vmatrix} \\
&= \abs{I_m-\phi_1 z -\phi_2 z^2 - \cdots \phi_p z^p} 
\end{align*}
Note that we iterative eliminate the off-diagonal entries in the lower triangle matrix(starting from the p-1 column, then p-2 column, until the first column).
\end{proof}


\begin{lemma}[invertibility of VAR(p) operator and MA representation]\cite[23]{lutkepohl2005new}
Consider the VAR(p) process given as
$$y_t = C + (A_1B + ... + A_pB^p)y_t + u_t$$
and define 
$$A(B) = I_K - A_1B - ... - A_pB^p.$$
We have
\begin{itemize}
\item If $det(I_K - A_1z - ... - A_p z^p) \neq 0, \forall \abs{z} \leq 1 $, then $A(B)$ is invertible.
\item If $A(B)$ is invertible, let $\Phi(B)$ denote its inverse such that $\Phi(B)A(B) = I_K$, then 
$$\Phi(B) = \sum_{i=1}^\infty \Phi_i B^i$$
where
\begin{align*}
I_K &= \Phi_0\\
0 &=\Phi_1 - \Phi_0A_1\\
0 &=\Phi_2 - \Phi_1A_1 - \Phi_0A_2\\
\vdots & \\
0 &=\phi_i - \sum_{j=1}^i \Phi_{i-j}A_j
\vdots &
\end{align*}
\end{itemize}
\end{lemma}

\begin{corollary}[ MA representation]
A stable $VAR(p)$ process has its moving average representation as
$$y_t = \Phi(B)C + \Phi(B)u_t$$
where $\Phi(B) = (I_K - A_1B - ... - A_pB^p)^{-1}$.
\end{corollary}


\begin{remark}[connection with Wold decomposition]
The MA representation is an example of the Wold decomposition for stationary multivariate time series(\autoref{ch:time-series-analysis:th:wolddecompositionmultiD}).
\end{remark}


\begin{lemma}[mean of VAR(p), and mean-adjusted form]
For a stable Var(p) process $X_t$, we have
\begin{itemize}
\item $\mu = \Phi(B)C$, where $\Phi(B) = (I_K - A_1B - ... - A_pB^p)^{-1}$.
\item It can be written as
$$X_t - \mu = A_1(X_{t-1} - \mu) + A_2(X_{t-2} - \mu) + ... + A_p(X_{t-p} - \mu) + \eta_t.$$
\end{itemize} 
\end{lemma}

\subsection{Parameter estimation}


\section{Vector autoregressive moving-average(VMA) model}
\subsection{Vector moving-average model }
\begin{definition}[VMA operator]
A VAR operator of order $q$ is a matrix polynomial of lag operator given as
$$\Theta_p(B) = I_k - \Theta_1B - \Theta_2 B^2 - ... - \Theta_p B^p$$
where $B$ is the lag operator, $\Theta_i \in \R^{k\times k}$
\end{definition}


\begin{definition}[VMA(q) model]\cite[106]{tsay2013multivariate}
A VMA model of order $q$ for a m dimensional time series $z_t$ is given by
$$z_t = \mu + a_t - \sum_{i=1}^q \theta_i a_{t-i},$$
where
\begin{itemize}
	\item $\mu\in \R^m$ is a constant vector;
	\item $\theta_i \in \R^{m\times m}, i=1,2,...,q$ are matrices and $\theta_q \neq 0$;
	\item $\{a_t\}$ is a white noise process with $Var[a_t] = \Sigma_a \in \R^{m\times m}$. 
\end{itemize}

Using VMA operator, the VMA(q) model can also be written as
$$z_t = \mu + \Theta_q a_t.$$	
\end{definition}


\begin{lemma}[properties of VMA(q) model]\cite[110]{tsay2013multivariate}
Consider a VMA(q) model of a time series $z_t$. It follows that
\begin{itemize}
	\item $E[z_t] = \mu$.
	\item $Var[z_t] = \Gamma_0 = \Sigma_a + \sum_{i=1}^q \theta_i \Sigma_a \theta_i^T$.
	\item $$Cov(z_t, z_{t-j}) = \Gamma_j = \sum_{i=j}^q \theta_j \Sigma_a \theta_{i-j}^T,$$
	where $\theta_0 = -I_k, \forall j=1,...,q$.
	\item $\Gamma_j  \triangleq Cov(z_t, z_{t-j}) = 0, \forall j > q$.
\end{itemize}
\end{lemma}
\begin{proof}
(1) $$E[z_t] = E[\mu + a_t - \sum_{i=1}^q \theta_i a_{t-i}] = \mu + 0.$$
(2)	
\begin{align*}
Var[z_t] &= E[(z_t - \mu)(z_t - \mu)^T] \\
& =  E[\Theta_qa_t a_t^T\Theta_q^T] \\
& = E[a_t a_t^T] + \sum_{i=1}^q \theta_i \Sigma_a \theta_i^T 
\end{align*}
(3)(4) Note that
\begin{align*}
Var[z_t] &= E[(z_t - \mu)(z_t - \mu)^T] \\
& =  E[\Theta_q B^j a_t a_t^T\Theta_q^T] 
\end{align*}
\end{proof}


\subsection{Vector ARMA model}

\begin{definition}[VARMA(p,q) model]\cite[127]{tsay2013multivariate}
A m-dimensional time series $z_t$ is a VARMA(p,q) process if
$$\Phi_p z_t = C + \Theta_q a_t,$$
where
\begin{itemize}
	\item $C \in \R^m$ is a constant vector;
	\item $\Phi_p = I_m - \sum_{i=1}^p \phi_i B^i, \phi_i\in \R^{m\times m}$;
	\item $\Theta_p = I_m - \sum_{i=1}^q \theta_i B^i, \theta_i \in \R^{m\times m}$;
	\item $\{a_t\}$ is a white noise process with $Var[a_t] = \Sigma_a \in \R^{m\times m}$. 	
\end{itemize}	
\end{definition}


\begin{remark}
identifiability issue (TODO) \cite[128]{tsay2013multivariate}.
\end{remark}

\section{Structural analysis}

\begin{definition}[Granger casualty based on linear predictor]\cite[42]{lutkepohl2005new}\index{Granger causalty}
Let $\Omega_t = \{z_s,x_s|s\leq t\}$, let $T(z_{t+h}|\Omega_t)$ be the optimal linear prediction of $z_{t+h}$ based on $\Omega_t$. Then $x_t$ is said to Granger cause $z_t$ if the MSE of the predictor $T(z_{t+h}|\Omega_t)$ is less than the MSE of the predictor $T(z_{t+h}|\Omega_{t} \\ \{x_s|s\leq t\}\})$.
\end{definition}


\begin{remark}[interpretation]\hfill
\begin{itemize}
\item If a variable, or a group of variables, $y_1$ is found to be helpful for predicting another variable, or group of variables $y_2$, then $y_1$ is said to Granger-cause $y_2$.
\item The notation of Granger causality does not imply true causality. Granger causality is a statistical concept of causality that is based on prediction. 
\end{itemize}
\end{remark}



\section{Cointegration}
\subsection{Concepts and basics}
\subsubsection{order of integration}
\begin{definition}[integrated of order]\index{integrated of order}\cite[559]{hayashi2000econometrics}
An $m$-dimensional time series $\bm{X}_t = (x_{1,t},...,x_{m,t})$ is \textbf{integrated of order }$d$,denoted by $I(d)$, if
for each component 
$$\nabla^d \bm{X}_t = (1-L)^d\bm{X}_t$$
is stationary.
\end{definition}

\begin{definition}[integrated of order 0 process]\cite[563]{hayashi2000econometrics}
A time series $X_t$ is integrated of order 0, denoted $I(0)$, if it admits a moving average representation 

with 
$$\sum_{k=0}^\infty \abs{b_k^2} < \infty$$
\end{definition}

\begin{definition}[integrated of order 0 process, alternative]
A time series $X_t$ is integrated of order 0, denoted $I(0)$, if it admits a moving average representation with 
$$\sum_{k=0}^\infty \abs{b_k^2} < \infty$$
\end{definition}

\begin{remark}[interpretation]
This implies that the autocovariance is decaying to 0 sufficiently quickly. This is a necessary, but not sufficient condition for a stationary process. Therefore, all stationary processes are I(0), but not all I(0) processes are stationary.
\end{remark}

\begin{lemma}[integration order of AR(p) process ]\cite[238]{lutkepohl2005new}
Consider an AR(p) process $\Phi_p X_t = Z_t$. If its characteristic polynomial has $m$ unit root and the rest of roots lying outside the unit disk, then it is $I(m)$.
\end{lemma}
\begin{proof}
Note that $\Phi_p^{-1}$ contains $m$ factors $(I-L)^{-1}$. When we apply $(1-L)^m$ to $\Phi^{-1}$, 
$$X_t = (1-L)^m \Phi_p^{-1} Z_t$$
 will be a stationary process.  
\end{proof}

\begin{example}
Consider $X_t$ is given by
$$X_t = X_{t-1} + \epsilon_t,\epsilon_t \in WN(0,1).$$

$X_t$ is integrated of order 1 since 
$$X_t - X_{t-1} = \epsilon_t \sim I(0).$$	
\end{example}



\begin{lemma}[construct high order integrated process]
Given a $I(d-1)$ time series $X_t$, the process
$$Z_t = \sum_{k=0}^t X_k$$
is $I(d)$.
\end{lemma}
\begin{proof}
$\nabla Z_t =  \sum_{k=0}^t \nabla X_k = X_t \sim I(d-1)$
\end{proof}




\subsubsection{cointegration}
\begin{definition}[cointegration]\index{cointegration}
Consider an $m$-dimensional time series $\bm{X}_t = (x_{1,t},...,x_{m,t})$ of integrated of order 1. $\bm{X}_t$ is called cointegrated if there exists $\beta \in \R^m$ such that
$$\beta^T \bm{X_t}$$
is a stationary process.
\end{definition}

\begin{example}
For instance, a stock market index and the price of its associated futures contract move through time, each roughly following a random walk. Testing the hypothesis that there is a statistically significant connection between the futures price and the spot price could now be done by testing for the existence of a cointegrated combination of the two series.
\end{example}

\begin{remark}[cointegration vs. correlation]
Consider two time series $X_1(t)$ and $X_2(t)$, 
\begin{itemize}
\item The correlation specifies co-movement of fluctuations around the mean (i.e., $X_1(t)-E[X_1]$ and $X_2(t)-E[X_2]$); the correlation tends to be a short-term relationship.
\item The cointegration specifies co-movement of values,i.e., $X_1(t)$ and $X_2(t)$; the cointegration tends to be a long-term relationship.
\end{itemize}
\end{remark}



\begin{definition}[trend stationary process]\index{trend stationary process}
A process $Y_t$ is said to be trend stationary if
$$Y_t = f(t) + e_t$$
where $t$ is the time index, $f$ is the function mapping from time index to $\R$, and $e_t$ is a stationary process.
\end{definition}

\begin{example}\hfill
\begin{itemize}
\item $Y_t = at + b +e_t$ is a stationary process with linear trend.
\item $Y_t = e^t + e_t$ is a stationary process with exponential trend.
\end{itemize}
\end{example}

\subsubsection{Test of cointegration}



\subsubsection{Cointegration parameter estimation}





\subsection{Vector error correction model}


\begin{definition}[vector error correction model]\cite[248]{lutkepohl2005new}
Denote $\Delta y_t = y_t - y_{t-1}$. Let $w_t$ be a $K$-dimensional Brownian motion, $A_i\in \R^{K\times K}$	
\begin{itemize}
	\item Let $y_t$ be a $K$-dimensional VAR(1) process, given by
	$$y_t = A_1y_{t-1} + w_t$$
	
	subtracting $y_{t-1}$ on both sides, we get the VECM as
	
	$$\Delta y_t = \Pi y_{t-1} + w_t$$
	where
	$$\Pi = - (I_K - A_1)$$
	\item Let $y_t$ be a $K$-dimensional VAR(2) process, given by
	$$y_t = A_1y_{t-1} + A_2y_{t-2} + w_t$$
	
	The VECM is given by
	
	$$\Delta y_t = \Pi y_{t-1} + \Gamma\Delta y_{t-1} + w_t$$
	where
	$$\Pi = - (I_K - A_1), \Gamma = -A_{2}.$$
	\item 	Let $y_t$ be a $K$-dimensional VAR(p) process, given by
	$$y_t = A_1y_{t-1} + \cdots + A_py_{t-p} + w_t.$$
	
	The VECM is given by
	
	$$\Delta y_t = \Pi y_{t-1} + \Gamma_1\Delta y_{t-1} + \cdots + \Gamma_{p-1}\Delta y_{t-p+1} + w_t$$
	where
	$$\Pi = - (I_K - A_1 - \cdots - A_p)$$
	and
	$$\Gamma_i = -(A_{i+1}+\cdots + A_p), i=1,...,p-1$$
\end{itemize}	 
\end{definition}

\begin{remark}[verification of the equivalence]
To verify the equivalence of VECM to the original form, we have(take VaR(2) as an example)
\begin{align*}
\Delta y_t &= \Pi y_{t-1} + \Gamma\Delta y_{t-1} + w_t \\
y_t - y_{t-1} &= (-I_k+A_1+A_2)y_{t-1} - A_2(y_{t-1}-y_{t-2})+w_t \\
y_t &= A_1y_{t-1} + A_2y_{t-2} + w_t
\end{align*}	
\end{remark}


\begin{lemma}[identify cointegration vector vector error correction form]\cite[248]{lutkepohl2005new}
Let $y_t$ be a $K$-dimensional VAR(p) process, given by
$$y_t = A_1y_{t-1} + \cdots + A_py_{t-p} + w_t$$
with
$$\Delta y_t = \Pi y_{t-1} + \Gamma_1\Delta y_{t-1} + \cdots + \Gamma_{p-1}\Delta y_{t-p+1} + w_t$$
where
$$\Pi = - (I_K - A_1 - \cdots - A_p)$$
and
$$\Gamma_i = -(A_{i+1}+\cdots + A_p), i=1,...,p-1$$

Let $rank(\Pi) = r$.
It follows that
\begin{itemize}
	\item If $r = 0$, or equivalently $\Pi = 0$, then $y_t$ is not cointegrated.
	\item if $0<r <K$ and let the singular decomposition of $\Pi$ be $\Pi = UV^T$, where $U\in \R^{K\times m}, V\in \R^{K\times m}$, and each column in $V$ is one possibility of cointegration vector. 
	\item if $r=K$, then $y_t$ is not unit-root process.
\end{itemize}


\end{lemma}
\begin{proof}	
Because the left-hand side and the right-hand side are both $I(0)$(stationary), we must have $\Pi y_{t-1}$ to be $I(0)$(otherwise, suppose $\Pi y$ is not, then the equality cannot hold).	
Suppose the of $rank(\Pi) = m < K$, then based on the singular value theory(\autoref{ch:linearalgebra:th:SVD}), we can decompose $\Pi = UV^T$, where $U\in \R^{K\times m}, V\in \R^{K\times m}$, and each column in $V$ is one possibility of cointegration vector. 

\end{proof}


\begin{note}[non-uniqueness of cointegration vector]\cite[249]{lutkepohl2005new}
\end{note}

\section{State space models}
\begin{definition}[state space model]\index{state space model}
A state space model is given as
\begin{align*}
z_t &= A_tz_{t-1} + \epsilon_t\\
y_t &=C_tz_t + \delta_t
\end{align*}
where $z_t\in \R^d,A\in \R^{d\times d}, \epsilon_t \sim MN(0,Q), Q\in \R^{d\times d}, y_t \in \R^p, C_t\in \R^{p\times d}, \delta_t \sim MN(0,R).$
If $A_t, C_t$ are indepedent of time $t$, then it is called time-invariant state space model.
\end{definition}


\begin{remark}[non-uniqueness of representation]
Consider any nonsingular matrix $M\in \R^{d\times d}$, then
\begin{align*}
Mz_t = MA_tM^{-1}Mz_{t-1} + M\epsilon_t &\implies z_t^* = A_t^* z_{t-1}^* + \epsilon_t^*\\
y_t = C_tM^{-1}Mz_{t-1} + \delta_t &\implies y_t = G_t^* z_{t}^* + \delta_t\\
\end{align*}
That is, $y_t$ will still be the same.
\end{remark}

\begin{remark}[dimension of state and minimal dimension]
One can increase the dimension of the state vector without 'adding information'. For example, let $M\in \R^{k\times d},k>d$($M$ is required to have full column rank), and let $M* = (M^TM)^{-1}M^T \in \R^{d\times k}$ be its pseudo inverse, then
 \begin{align*}
 Mz_t = MA_tM^*Mz_{t-1} + M\epsilon_t &\implies z_t^* = A_t^* z_{t-1}^* + \epsilon_t^*\\
 y_t = C_tM^*Mz_{t-1} + \delta_t &\implies y_t = G_t^* z_{t}^* + \delta_t\\
 \end{align*}
That is, we increase the dimensionality of the state vector $z_t$ to higher dimension; however, $y_t$ remains the same.\\
\textbf{note:}
If $k<d$ or $M$ does not have full column rank, then 
$(M^TM)^{-1}$ does not exist due to \autoref{ch:linearalgebra:ranklemmaone}. Then we cannot do such transformation without losing information.
\end{remark}


\section{Notes on Bibliography}


Introductory level treatment, see \cite{chatfield2003analysis}.
Intermediate level treatment, see \cite{brockwell2002introduction}\cite{hamilton1994time}\cite{shumway2010time}.
Advanced level treatment, see \cite{brockwell1991time}


For multivariate time series and cointegration, see \cite{tsay2013multivariate}\cite{pfaff2008analysis}\cite{lutkepohl2005new}.

For finanical time series, see \cite{tsay2005analysis}\cite{tsay2013multivariate}\cite{hayashi2000econometrics}.


Analysis of Integrated and Cointegrated Time Series with R


\printbibliography